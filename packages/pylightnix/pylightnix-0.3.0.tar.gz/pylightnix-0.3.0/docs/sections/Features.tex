\section{Features}

We tried to meet high development standards. In particular, Pylightnix:
\begin{itemize}
  \item Is written in Python 3.7. \href{http://mypy-lang.org/}{Mypy} typing is
    used whenever possible.
  \item Is tested using \href{https://pypi.org/project/hypothesis/}{Hypothesis}.
  \item Documentation is written in a
    \href{https://en.wikipedia.org/wiki/Literate_programming}{literate
    programming} style. Most of the code examples shown in this manual were
    checked and evaluated inline by
    \href{https://github.com/gpoore/pythontex}{PythonTex} tool.
  \item Core modules depend solely on the Python standard library. Optional
    modules do depend on \href{https://curl.se/}{Curl} and
    \href{https://www.gnu.org/software/wget/}{Wget} for accessing the Internet
    and on \href{https://www.nongnu.org/atool/}{Atool} to deal with compressed
    files.
  \item Alas, Pylightnix is not a production-ready yet! No means of parallelism
    are provided, network synchronization is yet under development.  We didn't
    check Pylightnix on any operating system besides Linux. We tried our best
    to make Pylightnix' storage operations atomic. Among other benefits,
    this allows running several instances of the library on a single
    storage at once.
\end{itemize}

