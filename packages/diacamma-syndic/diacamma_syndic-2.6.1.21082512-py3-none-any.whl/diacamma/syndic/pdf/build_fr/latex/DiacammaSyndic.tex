%% Generated by Sphinx.
\def\sphinxdocclass{report}
\documentclass[a4paper,10pt,oneside,french]{sphinxmanual}
\ifdefined\pdfpxdimen
   \let\sphinxpxdimen\pdfpxdimen\else\newdimen\sphinxpxdimen
\fi \sphinxpxdimen=.75bp\relax
\ifdefined\pdfimageresolution
    \pdfimageresolution= \numexpr \dimexpr1in\relax/\sphinxpxdimen\relax
\fi
%% let collapsable pdf bookmarks panel have high depth per default
\PassOptionsToPackage{bookmarksdepth=5}{hyperref}

\PassOptionsToPackage{warn}{textcomp}
\usepackage[utf8]{inputenc}
\ifdefined\DeclareUnicodeCharacter
% support both utf8 and utf8x syntaxes
  \ifdefined\DeclareUnicodeCharacterAsOptional
    \def\sphinxDUC#1{\DeclareUnicodeCharacter{"#1}}
  \else
    \let\sphinxDUC\DeclareUnicodeCharacter
  \fi
  \sphinxDUC{00A0}{\nobreakspace}
  \sphinxDUC{2500}{\sphinxunichar{2500}}
  \sphinxDUC{2502}{\sphinxunichar{2502}}
  \sphinxDUC{2514}{\sphinxunichar{2514}}
  \sphinxDUC{251C}{\sphinxunichar{251C}}
  \sphinxDUC{2572}{\textbackslash}
\fi
\usepackage{cmap}
\usepackage[T1]{fontenc}
\usepackage{amsmath,amssymb,amstext}
\usepackage{babel}



\usepackage{tgtermes}
\usepackage{tgheros}
\renewcommand{\ttdefault}{txtt}



\usepackage[Sonny]{fncychap}
\ChNameVar{\Large\normalfont\sffamily}
\ChTitleVar{\Large\normalfont\sffamily}
\usepackage{sphinx}

\fvset{fontsize=auto}
\usepackage{geometry}


% Include hyperref last.
\usepackage{hyperref}
% Fix anchor placement for figures with captions.
\usepackage{hypcap}% it must be loaded after hyperref.
% Set up styles of URL: it should be placed after hyperref.
\urlstyle{same}


\usepackage{sphinxmessages}
\setcounter{tocdepth}{3}
\setcounter{secnumdepth}{3}


\title{Diacamma Syndic}
\date{août 25, 2021}
\release{2.6.1}
\author{sd-libre}
\newcommand{\sphinxlogo}{\sphinxincludegraphics{DiacammaSyndic.jpg}\par}
\renewcommand{\releasename}{Version}
\makeindex
\begin{document}

\ifdefined\shorthandoff
  \ifnum\catcode`\=\string=\active\shorthandoff{=}\fi
  \ifnum\catcode`\"=\active\shorthandoff{"}\fi
\fi

\pagestyle{empty}
\sphinxmaketitle
\pagestyle{plain}
\sphinxtableofcontents
\pagestyle{normal}
\phantomsection\label{\detokenize{index::doc}}



\chapter{Diacamma Syndic}
\label{\detokenize{syndic/index:diacamma-syndic}}\label{\detokenize{syndic/index::doc}}
\sphinxAtStartPar
Présentation du logiciel Diacamma Syndic.


\section{Présentation}
\label{\detokenize{syndic/presentation:presentation}}\label{\detokenize{syndic/presentation::doc}}

\subsection{Description}
\label{\detokenize{syndic/presentation:description}}
\sphinxAtStartPar
\sphinxstyleemphasis{Diacamma Syndic} est un logiciel de gestion spécialement conçu pour les copropriétés et les syndics bénévoles.
Avec \sphinxstyleemphasis{Diacamma}, donnez à votre structure le logiciel qu’elle mérite ! Pas besoin d’être informaticien pour avoir les outils adaptés à votre cas.

\sphinxAtStartPar
L’application de base est entièrement gratuite et vous permet de gérer les accès à vos données et au carnet d’adresses nécessaire à votre copropriété.

\sphinxAtStartPar
Les différents modules disponibles vous permettront, par exemple, de:
\begin{itemize}
\item {} 
\sphinxAtStartPar
Identifier les copropriétaires et leurs différents tantièmes, réaliser des appels de fonds correspondant à votre budget, ventiler les factures et les frais de la copropriété suivant les parts de chacun.

\item {} 
\sphinxAtStartPar
Gérer vos documents de façon centralisée grâce à la gestion documentaire.

\item {} 
\sphinxAtStartPar
Gérer une comptabilité simple et performante et clôturer rapidement votre exercice comptable pour votre assemblée générale.

\end{itemize}

\sphinxAtStartPar
Ce manuel vous aidera dans l’utilisation de ce logiciel.
Si malgré tout, vous ne trouvez pas la réponse à vos problèmes, visiter notre site \sphinxurl{https://www.diacamma.org} où vous trouverez des tutoriels et des astuces.


\subsection{Installation}
\label{\detokenize{syndic/presentation:installation}}
\sphinxAtStartPar
Vous pouvez installer \sphinxstyleemphasis{Diacamma Syndic} sur un ordinateur dédié à votre copropriété, que ce soit un Apple Macintosh (macOS 10.12 « Sierra » et +) ou bien un PC sous MS\sphinxhyphen{}Windows (8 et +) ou sous GNU Linux (Ubuntu 16.04 ou +).

\sphinxAtStartPar
\sphinxstyleemphasis{Diacamma Syndic} est un logiciel client/serveur : vous pouvez l’installer sur un ordinateur centralisateur et accéder aux données depuis un autre PC connecté au premier, sans limite du nombre d’utilisateurs simultanés.
Si le PC contenant les données est connecté de manière permanente à internet, vous aurez accès à vos données depuis n’importe où dans le monde !

\sphinxAtStartPar
Cette organisation est particulièrement intéressante pour permettre à plusieurs intervenants d’avoir accès à des données communes.

\sphinxAtStartPar
Quel responsable ne s’est pas arraché les cheveux suite à un échange de documents via une clef USB où certaines modifications importantes se perdent ?

\sphinxAtStartPar
Pour plus d’information, visiter notre site \sphinxurl{https://www.diacamma.org}


\subsection{Aides et support}
\label{\detokenize{syndic/presentation:aides-et-support}}
\sphinxAtStartPar
Sur le site officiel du logiciel, \sphinxurl{https://www.diacamma.org}, vous trouverez des tutoriels et un forum pour échanger des astuces entre utilisateurs.

\sphinxAtStartPar
Si vous souhaitez un service et un support plus personnalisé, vous pouvez faire confiance à notre partenaire officiel SLETO.
Pour en savoir plus sur des solutions d’hébergement et de support, rendez vous sur \sphinxurl{https://www.sleto.net}


\section{Prise en main}
\label{\detokenize{syndic/first_step:prise-en-main}}\label{\detokenize{syndic/first_step::doc}}
\sphinxAtStartPar
Le logiciel \sphinxstyleemphasis{Diacamma Syndic} comprend un grand nombre de paramétrages et peut paraître difficile à configurer pour couvrir les besoins de votre structure.

\sphinxAtStartPar
Nous vous proposons ce guide pour vous aider à franchir cette première étape dans l’utilisation de cet outil.

\sphinxAtStartPar
Suivez pas à pas les différents phases de réglage. Dans chaque étape, nous ne détaillons pas les fonctionnalités. Nous vous invitons également à vous référer au reste du manuel utilisateur pour cela.

\sphinxAtStartPar
Il peut être intéressant de réaliser des sauvegardes au cours de cette procédure.
Cela vous permettra, si vous faites une erreur, de revenir à une étape précédente sans tout recommencer (installation comprise).


\subsection{Vérifiez la mise à jour de votre logiciel}
\label{\detokenize{syndic/first_step:verifiez-la-mise-a-jour-de-votre-logiciel}}
\sphinxAtStartPar
Commencez par vérifier que votre logiciel est à jour.
En effet, nous diffusons régulièrement des correctifs qui ne sont pas toujours inclus dans les installateurs.


\subsection{Présentation de votre copropriété}
\label{\detokenize{syndic/first_step:presentation-de-votre-copropriete}}\begin{quote}

\sphinxAtStartPar
Menu \sphinxstyleemphasis{General/Nos coordonnées}
\end{quote}

\sphinxAtStartPar
Dans cet écran, vous pouvez décrire les coordonnées de votre structure.
De nombreuses fonctionnalités utilisent ces informations en particulier pour les impressions.


\subsection{Création de votre premier exercice comptable}
\label{\detokenize{syndic/first_step:creation-de-votre-premier-exercice-comptable}}\begin{quote}

\sphinxAtStartPar
Menu \sphinxstyleemphasis{Administration/Modules (conf.)/Configuration comptable}

\sphinxAtStartPar
Menu \sphinxstyleemphasis{Comptabilité/Gestion comptable/Plan comptable} \sphinxhyphen{} onglet « Configuration comptable »
\end{quote}

\sphinxAtStartPar
Ouvrez votre premier exercice comptable et rendez\sphinxhyphen{}le actif.
Vous devrez aussi créer le plan comptable de cet exercice pour avoir une comptabilité pleinement opérationnelle.


\subsection{Réglage de votre copropriété}
\label{\detokenize{syndic/first_step:reglage-de-votre-copropriete}}\begin{quote}

\sphinxAtStartPar
Menu \sphinxstyleemphasis{Administration/Modules (conf.)/Configuration de copropriété}
\end{quote}

\sphinxAtStartPar
Vérifiez que ces paramètres correspondent à votre utilisation.


\subsection{Définition des catégories de charges, des lots et des tantièmes}
\label{\detokenize{syndic/first_step:definition-des-categories-de-charges-des-lots-et-des-tantiemes}}\begin{quote}

\sphinxAtStartPar
Menu \sphinxstyleemphasis{Copropriété/Les propriétaires et les lots}

\sphinxAtStartPar
Menu \sphinxstyleemphasis{Copropriété/Les catégories de charges}
\end{quote}

\sphinxAtStartPar
Déclarez vos copropriétaires ainsi que les lots de votre copropriété.
Ajoutez vos catégories de charges et définissez pour chacune les quote\sphinxhyphen{}parts ou les lots associés.


\subsection{Mise à jour comptable}
\label{\detokenize{syndic/first_step:mise-a-jour-comptable}}\begin{quote}

\sphinxAtStartPar
Menu \sphinxstyleemphasis{Comptabilité/Gestion comptable/Écritures comptables}

\sphinxAtStartPar
Menu \sphinxstyleemphasis{Comptabilité/Gestion comptable/Modèles d’écriture}
\end{quote}

\sphinxAtStartPar
Si vous mettez en place \sphinxstyleemphasis{Diacamma Syndic} en cours d’exercice comptable, vous devrez dans un premier temps saisir les à\sphinxhyphen{}nouveaux et le report à nouveau (ASL) de l’exercice précédent ainsi que saisir les écritures du début d’année.

\sphinxAtStartPar
Pour vous aidez dans la saisie de votre comptabilité, nous vous conseillons d’utiliser les modèles d’écriture. Enregistrez en tant que modèles les écritures récurrentes que vous passez plusieurs fois au cours d’une année. Ainsi vous pouvez rapidement compléter votre comptabilité en quelques clics.


\subsection{Le courriel}
\label{\detokenize{syndic/first_step:le-courriel}}\begin{quote}

\sphinxAtStartPar
Menu \sphinxstyleemphasis{Administration/Modules (conf.)/Paramètrages de courriel}
\end{quote}

\sphinxAtStartPar
Définissez vos réglages pour votre courriel.
Le serveur smpt permettra à \sphinxstyleemphasis{Diacamma Syndic} d’envoyer un certain nombre de messages : documents en PDF, mot de passe de connexion, …

\sphinxAtStartPar
Vous pouvez préciser comment réagissent les liens “écrire à tous” avec votre logiciel de messagerie.


\subsection{Les responsables}
\label{\detokenize{syndic/first_step:les-responsables}}\begin{quote}

\sphinxAtStartPar
Menu \sphinxstyleemphasis{Administration/Modules (conf.)/Configuration des contacts}

\sphinxAtStartPar
Menu \sphinxstyleemphasis{Bureautique/Adresses et contacts/Personnes…}

\sphinxAtStartPar
Menu \sphinxstyleemphasis{Général/Nos coordonnées}
\end{quote}

\sphinxAtStartPar
Dans la fenêtre de vos coordonnées, vous pouvez associer les personnes responsables de votre syndic et de votre syndicat de copropriétaires.
Utilisez l’outil de recherche et assignez\sphinxhyphen{}leur une fonction.
Vous pouvez également rajouter des fonctions propres à votre structure.
\begin{quote}

\sphinxAtStartPar
Menu \sphinxstyleemphasis{Administration/Modules (conf.)/Les groupes}

\sphinxAtStartPar
Menu \sphinxstyleemphasis{Administration/Modules (conf.)/Utilisateurs du logiciel}/Nos coordonnées*
\end{quote}

\sphinxAtStartPar
Privilégiez une utilisation du logiciel avec un alias et un mot de passe propres à chaque utilisateur. Associez\sphinxhyphen{}leur également les droits correspondant à leurs fonctions au sein de votre structure.
Enfin, évitez autant que possible l’utilisation de l’alias “admin” qui doit être réservé pour des actions de configuration et de maintenance.


\subsection{La gestions documentaire}
\label{\detokenize{syndic/first_step:la-gestions-documentaire}}\begin{quote}

\sphinxAtStartPar
Menu \sphinxstyleemphasis{Administration/Modules (conf.)/Dossier}

\sphinxAtStartPar
Menu \sphinxstyleemphasis{Bureautique/Gestion de fichiers et de documents/Documents}
\end{quote}

\sphinxAtStartPar
Définissez vos différents dossiers dans lesquels seront enregistrés vos documents à classer et/ou à partager et/ou à imprimer.

\sphinxAtStartPar
Après avoir parcouru ces points, votre logiciel \sphinxstyleemphasis{Diacamma Syndic} est pleinement opérationnel.
N’hésistez pas à consulter le forum: de nombreuses astuces peuvent vous aider à utiliser au mieux votre logiciel.


\chapter{Diacamma copropriété}
\label{\detokenize{condominium/index:diacamma-copropriete}}\label{\detokenize{condominium/index::doc}}
\sphinxAtStartPar
Aide relative aux fonctionnalités de gestion de copropriété.


\section{Comptabilité : les contraintes réglementaires}
\label{\detokenize{condominium/account_contraints:comptabilite-les-contraintes-reglementaires}}\label{\detokenize{condominium/account_contraints::doc}}

\subsection{Une comptabilité d’engagement}
\label{\detokenize{condominium/account_contraints:une-comptabilite-d-engagement}}
\sphinxAtStartPar
Les opérations réalisées au titre de la copropriété doivent être comptabilisées par le syndic en respect du principe de la partie double,et ce dès l’engagement des dépenses et des recettes. Chaque exercice comptable étant indépendant, cela permet de lui rattacher les dépenses et les recettes qui lui sont propres sans qu’il soit tenu compte de la date effective des encaissements/décaissements. Toute copropriété doit établir un budget prévisionnel.

\sphinxAtStartPar
Le Livre\sphinxhyphen{}journal, obligatoire, enregistre chronologiquement les opérations ayant une incidence sur le fonctionnement du syndicat des copropriétaires.

\sphinxAtStartPar
Seules les petites copropriétés de moins de 10 lots et un budget prévisionnel inférieur à 15 000 \texteuro{} ne sont pas soumises à la comptabilité en partie double, mais doivent tout de même établir un budget prévisionnel.


\subsection{La durée de l’exercice comptable}
\label{\detokenize{condominium/account_contraints:la-duree-de-l-exercice-comptable}}
\sphinxAtStartPar
\sphinxstyleemphasis{Décret n°2005\sphinxhyphen{}240 du 14 mars 2005 relatif aux comptes du syndicat des copropriétaires \textendash{} article 5.}
\begin{quote}

\sphinxAtStartPar
L’exercice comptable du syndicat des copropriétaires couvre une période de douze mois. Les comptes sont arrêtés à la date
de clôture de l’exercice. Pour le premier exercice, l’assemblée générale des copropriétaires fixe la date de clôture des
comptes et la durée de cet exercice qui ne pourra excéder dix\sphinxhyphen{}huit mois.
La date de clôture de l’exercice pourra être modifiée sur décision motivée de l’assemblée générale des copropriétaires.
Un délai minimum de cinq ans devra être respecté entre les deux décisions d’assemblées générales modifiant la date
de clôture.
\end{quote}


\subsection{Plan comptable}
\label{\detokenize{condominium/account_contraints:plan-comptable}}
\sphinxAtStartPar
Le plan comptable de la copropriété doit respecter la nomenclature des comptes énoncée dans l’arrêté du 14 mars 2005 relatif aux comptes du syndicat des copropriétaires, avec possibilité de subdivision en cas de besoin (ex. 512 Banques \(\rightarrow\) 512100 Mabanque).

\sphinxAtStartPar
Le Plan Comptable Général des entreprises ne doit pas être utilisé.


\section{Les copropriétaires}
\label{\detokenize{condominium/owners:les-coproprietaires}}\label{\detokenize{condominium/owners::doc}}
\sphinxAtStartPar
La réglementation française impose d’individualiser les comptes des copropriétaires. Cela nécessite d’ouvrir une subdivision du  compte 450000 pour chaque copropriétaire. L’AG peut aussi décider de distinguer les créances selon leur nature (comptes “450\sphinxhyphen{}100”, “450\sphinxhyphen{}200”, “450\sphinxhyphen{}300”…).

\sphinxAtStartPar
Avant de créer vos copropriétaires, vous devez paramétrer votre copropriété en spécifiant les comptes généraux devant être utilisés par vous, grâce à quoi les subdivisions seront possibles :
\begin{quote}

\sphinxAtStartPar
Menu \sphinxstyleemphasis{Administration/Modules (conf.)/Configuration de la copropriété}
\end{quote}

\sphinxAtStartPar
Tout copropriétaire est un tiers, personne physique ou morale :
\begin{quote}

\sphinxAtStartPar
Menu \sphinxstyleemphasis{Bureautique/Adresses et contacts/Personnes…}
\end{quote}

\sphinxAtStartPar
qui est propriétaire de lot(s) de copropriété.
\begin{quote}

\sphinxAtStartPar
Menu \sphinxstyleemphasis{Copropriété/Gestion/Les propriétaires et les lots} \sphinxhyphen{} onglet « Les propriétaires »
\end{quote}

\sphinxAtStartPar
Ajoutez un propriétaire à la liste à l’aide du bouton « + Créer ».
Sélectionnez\sphinxhyphen{}le dans la liste des tiers. Cette liste est actualisable en utilisant le bouton « + ». Au préalable, assurez\sphinxhyphen{}vous que vos tiers sont bien existants.

\sphinxAtStartPar
Comptablement, un copropriétaire est également un tiers comptable pour lequel des comptes « copropriétaire individualisé » sont ouverts automatiquement, à la condition que votre copropriété ait été correctement paramétrée. Contrôlez les comptes généraux qui leur sont affectés.

\sphinxAtStartPar
La liste des copropriétaires affichée, vous pouvez aussi avoir le résumé de la situation financière de chacun.


\subsection{Les lots}
\label{\detokenize{condominium/owners:les-lots}}\begin{quote}

\sphinxAtStartPar
Menu \sphinxstyleemphasis{Copropriété/Gestion/Les propriétaires et les lots} \sphinxhyphen{} onglet « Les lots »
\end{quote}

\sphinxAtStartPar
Un lot peut être détenu par un ou plusieurs propriétaires et un propriétaire peut être propriétaire d’un ou plusieurs lots.

\sphinxAtStartPar
Déclarez les lots de votre copropriété.
Précisez en plus de la description du lot, les tantièmes et le propriétaire de chacun.
Vous devez veillez à la cohérence des tantièmes attribués aux lots sans quoi la répartition des charges et la détermination des appels de fonds seront incorrects.
\begin{quote}

\noindent\sphinxincludegraphics{{set_owner}.png}
\end{quote}


\section{Les catégories de charges}
\label{\detokenize{condominium/classloads:les-categories-de-charges}}\label{\detokenize{condominium/classloads::doc}}\begin{quote}

\sphinxAtStartPar
Menu \sphinxstyleemphasis{Copropriété/Gestion/Les catégories de charges}
\end{quote}

\sphinxAtStartPar
Vous pouvez individualiser la répartition des charges qui sont liées ou non aux lots.
Par exemple, vous pouvez distinguer comme catégories de charges : les montées d’excalier, les garages, le chauffage, l’eau, …


\subsection{Charges courantes et charges exceptionnelles}
\label{\detokenize{condominium/classloads:charges-courantes-et-charges-exceptionnelles}}
\sphinxAtStartPar
Une catégorie de charges donnée concerne soit des charges courantes, soit des charges exceptionnelles.
Les premières, en lien avec le fonctionnement et le maintien en bon état de la copropriété, sont récurrentes d’une année sur l’autre et doivent obligatoirement être budgétées (quelle que soit la taille de la copropriété) avant d’être engagées. Les secondes sont relatives à des opérations ponctuelles ne relevant pas du budget prévisionnel (ex. travaux votés en AG) pour lesquelles les catégories correspondantes seront clôturées une fois ces opérations terminées. Généralement, elles aussi font l’objet d’un budget bien que la réglementation ne l’impose pas.

\sphinxAtStartPar
La réglementation française impose de distinguer les charges de copropriété selon leur nature (charges courantes et charges exceptionnelles).


\subsection{La répartition}
\label{\detokenize{condominium/classloads:la-repartition}}
\sphinxAtStartPar
Deux modes de répartition peuvent être utilisés pour chaque catégorie de charges:
\begin{itemize}
\item {} 
\sphinxAtStartPar
Liée au lots

\sphinxAtStartPar
Dans ce cas, assignez un ensemble de lots sur cette catégorie. Pour cela, affichez la catégorie et cliquez sur le bouton « Editer ». Sélectionnez les lots concernés par la catégorie en les faisant passer de la liste de gauche sur la liste de droite.
Les tantièmes correspondants seront automatiquement associés aux différents propriétaires afin de définir la répartition des charges de la catégorie.

\item {} 
\sphinxAtStartPar
Non liée au lots

\sphinxAtStartPar
Vous devez, dans ce cas, définir pour les propriétaires concernés les tantièmes grâce auxquels les charges de la catégorie seront réparties.  Ceci fait, validez votre saisie et vérifiez l’association \sphinxstyleemphasis{Catégorie \sphinxhyphen{} Propriétaires concernés}.

\sphinxAtStartPar
Ce mode de répartition est utilisé dans le cas de consommations fixes (surface, étage, …) ou individuelles (consommation d’eau).

\end{itemize}


\subsection{Les budgets prévisionnels}
\label{\detokenize{condominium/classloads:les-budgets-previsionnels}}
\sphinxAtStartPar
Depuis l’interface d’une catégorie de charges, vous pouvez ajouter un budget prévisionnel.
Cliquez simplement sur le bouton \sphinxstyleemphasis{Budget} depuis sa fiche.

\sphinxAtStartPar
L’interface vous permet alors d’ajouter des comptes de charges et de produits en les assortissant du solde prévisionnel. Pour cela, utilisez le bouton « + Ajouter » situé au bas de l’écran du budget. Attention : Tout budget doit être en équilibre.
Vous pouvez également importer les soldes des comptes de charges et de produits d’un exercice précédent.

\sphinxAtStartPar
Le budget prévisionnel est repris dans les rapports « Comptes de gestion… », ce qui permet de comparer le prévisionnel avec le réalisé.


\section{Les appels de fonds}
\label{\detokenize{condominium/call_of_funds:les-appels-de-fonds}}\label{\detokenize{condominium/call_of_funds::doc}}
\sphinxAtStartPar
Avant de commencer, assurez\sphinxhyphen{}vous que vos catégories de charges sont bien existantes et les dépenses correspondantes réparties entre vos différents copropriétaires.
\begin{quote}

\sphinxAtStartPar
Menu \sphinxstyleemphasis{copropriété/Gestion/Les appels de fonds}
\end{quote}

\sphinxAtStartPar
Depuis cette liste des appels de fonds, vous pouvez en créer un nouveau via le bouton « + Créer ».

\sphinxAtStartPar
Une fois précisés la date et le descriptif de cet appel, une nouvelle fiche d’appel est créée dans laquelle les différents détails de l’appel sont saisis à l’aide du bouton « + Ajouter ». Pour chacun, sélectionnez la catégorie de charges concernée et le montant appelé.
\begin{quote}

\noindent\sphinxincludegraphics{{call_of_funds}.png}
\end{quote}

\sphinxAtStartPar
Enfin, pour finaliser l’appel de fonds, cliquez sur « Valider ».
L’ensemble des copropriétaires se voient alors associés à une nouvelle fiche d’appel de fonds.
Le montant de chaque détail d’appel précédemment saisi est ventilé dans les fiches d’appel de fonds individuelles, en tenant compte du ratio de chaque copropriétaire.

\sphinxAtStartPar
Ces appels de fonds individuels, générés d’après le modèle d’impression par défaut, sont automatiquement sauvegardés (sauvegarde officielle) dans le \sphinxstyleemphasis{Gestionnaire de documents} au moment de la validation.
Lorsque vous voulez imprimer un appel, le document sauvegardé est alors par défaut téléchargé.
Vous pouvez aussi régénérer un nouveau PDF, avec un autre modèle d’impression. Par contre celui\sphinxhyphen{}ci comportera la mention « duplicata » en filigrane.


\section{Le suivi des copropriétaires}
\label{\detokenize{condominium/payoff:le-suivi-des-coproprietaires}}\label{\detokenize{condominium/payoff::doc}}
\sphinxAtStartPar
Suite aux appels de fonds, les copropriétaires règlent ce qu’ils doivent au syndic.
Pour saisir un règlement depuis la liste des copropriétaires, allez dans le menu \sphinxstyleemphasis{Copropriété/Gestion/Les proprietaires et les lots}
Ouvrez la fiche du propriétaire, onglet « Appels et règlemennts »
\begin{quote}

\noindent\sphinxincludegraphics{{payoff}.png}
\end{quote}

\sphinxAtStartPar
Cliquez sur le bouton « Règlement ».
Ajoutez alors un paiement en précisant la date, le montant et la référence de la pièce comptable (comme le numéro de chèque).
Ce réglement se ventillera alors automatiquement sur ses appels de fonds en privilégiant le plus ancien.

\sphinxAtStartPar
Si le copropriétaire a versé plus que ce que lui demandait les appels de fonds, ces montants seront tracés dans le tableau « règlements supplémentaires ».
Il sera alors possible de reventiller ces suppléments sur les appels de fonds prochains.
En fin d’exercice, les montants trop perçu sont alors automatiquement reporté l’année suivante.


\section{Les dépenses}
\label{\detokenize{condominium/expense:les-depenses}}\label{\detokenize{condominium/expense::doc}}\begin{quote}

\sphinxAtStartPar
Menu \sphinxstyleemphasis{Copropriéte/Gestion/Les dépenses}.
\end{quote}

\sphinxAtStartPar
Depuis cette liste des dépenses, vous pouvez gérer une dépense de copropriété et en créer une nouvelle via le bouton « Ajouter ».

\sphinxAtStartPar
Une fois précisés la date et le descriptif de cette dépense, cliquez sur « Ok » afin qu’une nouvelle fiche de dépense soit créée.
Dans la nouvelle fenêtre, indiquez le fournisseur concerné par la nouvelle dépense. Celui\sphinxhyphen{}ci doit être un tiers référencé pour ce qui est des dépenses. S’il ne figure pas dans la liste des contacts, ajoutez\sphinxhyphen{}le avec le bouton « + Créer ».
Ajoutez dans cette fiche les détails de dépense en spécifiant pour chacun : la catégorie de dépenses concernée, la désignation du détail, le compte de charges à mouvementer et le montant du détail.
\begin{quote}

\noindent\sphinxincludegraphics{{expense}.png}
\end{quote}

\sphinxAtStartPar
Une fiche de dépense peut prendre, pour votre gestion, un ensemble de status.
Avec ce schéma, vous pouvez comprendre comment les statuts peuvent s’enchainner :
\begin{quote}

\noindent\sphinxincludegraphics{{expenseworkflow}.png}
\end{quote}

\sphinxAtStartPar
Voici, ici, une description plus précise de ces status.
\begin{itemize}
\item {} 
\sphinxAtStartPar
En création
\begin{quote}

\sphinxAtStartPar
Dans cet état, la fiche peut être complétement modifiée : description, date, type (facture ou avoir fournisseur), tiers, détails.
Par contre, aucun règlement ne peut être saisi.
Au niveau comptable, aucune écriture n’est alors générée.
A cette étape, il est encore possible de supprimer la fiche.
\end{quote}

\item {} 
\sphinxAtStartPar
En attente
\begin{quote}

\sphinxAtStartPar
Proche de l’état précédent, par contre il n’est plus possible de supprimer la fiche et un numéro unique lui est attribuée.
Il n’est plus possible de changer son tiers et son type par contre les règlements peuvent être saisi.
Au niveau comptable, seule la saisi des règlements sont traduit en écriture comptable (au brouillard).
\end{quote}

\item {} 
\sphinxAtStartPar
Validé
\begin{quote}

\sphinxAtStartPar
Plus aucune modification n’est possible à part l’ajout de nouveaux règlements.
En comptabilité, des écritures relatif aux détails sont générées au brouillard.
\end{quote}

\item {} 
\sphinxAtStartPar
Cloturé
\begin{quote}

\sphinxAtStartPar
La fiche est totalement non modifiable.
En comptabilité, l’ensemble des écritures liées à cette dépense sont alors validées.
\end{quote}

\item {} 
\sphinxAtStartPar
Annulé
\begin{quote}

\sphinxAtStartPar
C’est un état de conservation d’historique.
Plus aucune modification n’est possible.
Il n’y a plus de règlement (supprimés au changement d’état)
Les écritures comptables sont également supprimées.
\end{quote}

\end{itemize}


\section{Les rapports}
\label{\detokenize{condominium/report:les-rapports}}\label{\detokenize{condominium/report::doc}}
\sphinxAtStartPar
Vous retrouvez ici les rapports correspondant aux quatre premières annexes de la réglementation française.
L’annexe 5 n’est pas proposée actuellement dans Diacamma Syndic. celle\sphinxhyphen{}ci correspondant aux dépenses en lien avec les travaux et les dépenses exceptionnelles non clôturés en fin d’exercice. Pour obtenir un tel rapport, un outil de gestion spécifique doit être mis en place.

\sphinxAtStartPar
Ces rapports sont sauvegardés automatiquement dans le \sphinxstyleemphasis{Gestionnaire de documents} au moment de la clôture de l’exercice.
Si vous voulez les imprimer ou les rééditer, il vous est proposé par défaut de télécharger les documents sauvegardés.
Vous pouvez aussi générer un nouveau PDF, en utilisant votre propre modèle.
Par contre celui\sphinxhyphen{}ci comportera la mention « duplicata » en filigrane.


\subsection{État financier}
\label{\detokenize{condominium/report:etat-financier}}
\sphinxAtStartPar
Ce rapport présente la situation financière de la copropriété et la situation des copropriétaires :
\begin{itemize}
\item {} 
\sphinxAtStartPar
La trésorerie

\item {} 
\sphinxAtStartPar
Les créances

\item {} 
\sphinxAtStartPar
Les provisions, avances et fonds

\item {} 
\sphinxAtStartPar
Les emprunts

\item {} 
\sphinxAtStartPar
Les dettes

\end{itemize}


\subsection{Compte de gestion générale}
\label{\detokenize{condominium/report:compte-de-gestion-generale}}
\sphinxAtStartPar
Ce rapport présente les recettes et les dépenses relatives à l’exercice comptable :
\begin{itemize}
\item {} 
\sphinxAtStartPar
Les dépenses et les produits courants réels

\item {} 
\sphinxAtStartPar
Les dépenses et les produits exceptionnels réels

\item {} 
\sphinxAtStartPar
Les dépenses et les recettes budgétés (budgets n et n+1) tels que votés par les copropriétaires en AG.

\end{itemize}


\subsection{Compte de gestion pour opérations courantes}
\label{\detokenize{condominium/report:compte-de-gestion-pour-operations-courantes}}
\sphinxAtStartPar
Ce rapport présente pour chaque catégorie de dépenses ouverte, les dépenses budgétées (n et n+1) et les dépenses
réelles.


\subsection{Compte de gestion pour opérations exceptionnelles}
\label{\detokenize{condominium/report:compte-de-gestion-pour-operations-exceptionnelles}}
\sphinxAtStartPar
Ce rapport présente pour chaque catégorie de dépenses exceptionnelles, les dépenses budgétées (n et n+1) au titre
des travaux votés en AG et les dépenses réelles.


\chapter{Diacamma comptabilité}
\label{\detokenize{accounting/index:diacamma-comptabilite}}\label{\detokenize{accounting/index::doc}}
\sphinxAtStartPar
Aide relative aux fonctionnalités comptables.


\section{Définitions}
\label{\detokenize{accounting/definition:definitions}}\label{\detokenize{accounting/definition::doc}}\begin{quote}

\sphinxAtStartPar
\sphinxstylestrong{Remarques:} Ce module comptable est proche d’une comptabilité type « entreprise », néanmoins il ne respecte pas certaines exigences légales et fiscales en la matière.
Ce module ne peut être utilisé que pour la tenue des comptes de structures gérées par des bénévoles. Il n’est pas possible de l’utiliser pour la tenue des comptes de structures commerciales, concurrentielles ou professionnelles.
Le représentant légale de la structure utilisant ce module doit s’assurer que sa comptabilité respecte alors la législation en vigueur de son pays.
\end{quote}


\subsection{Exercice comptable}
\label{\detokenize{accounting/definition:exercice-comptable}}
\sphinxAtStartPar
L’exercice comptable est la période qui sépare deux déterminations du résultat. Il commence par une ouverture des comptes et se termine par une clôture de ceux\sphinxhyphen{}ci.

\sphinxAtStartPar
Pour ce qui est de la durée, le responsable de la structure doit respecter la réglementation. Il doit aussi tenir compte des statuts (association) et du règlement de copropriété (syndic), tant pour ce qui est du premier exercice que des suivants. Pour ce qui est de la modification de la durée des exercices, là encore il doit se référer aux textes réglementaires et à ceux qui régissent la vie de la structure.


\subsection{Tiers comptable}
\label{\detokenize{accounting/definition:tiers-comptable}}
\sphinxAtStartPar
Un tiers comptable est une personne physique ou morale avec laquelle une entité va avoir des échanges monétaires ou matériels (clients, fournisseurs, salariés, administrations…).


\subsection{Journaux comptables}
\label{\detokenize{accounting/definition:journaux-comptables}}
\sphinxAtStartPar
Un journal comptable est un regroupement d’écritures comptables permettant de classer celles\sphinxhyphen{}ci par nature.

\sphinxAtStartPar
Des journaux par défaut vous sont proposés.

\sphinxAtStartPar
Deux d’entre eux ne doivent pas être supprimés car utilisés lors de la clôture de l’exercice comptable et l’ouverture des comptes :
\begin{itemize}
\item {} 
\sphinxAtStartPar
journal des opérations diverses : écriture de détermination du résultat de l’exercice (en ce qui concerne une copropriété, ce résultat est en principe nul)

\item {} 
\sphinxAtStartPar
journal « Report à\sphinxhyphen{}nouveaux » : reprise des soldes des comptes de bilan en début d’exercice comptable

\end{itemize}

\sphinxAtStartPar
En ce qui concerne les autres journaux, vous pouvez adapter la liste type en créant, modifiant ou supprimant ceux qui vous sont proposés.


\subsection{Ecritures comptables}
\label{\detokenize{accounting/definition:ecritures-comptables}}
\sphinxAtStartPar
Une écriture comptable est la traduction dans les comptes d’une opération réalisée par ou pour le compte de votre structure et ayant une incidence sur son patrimoine. En respect du principe de la partie double, il y a égalité débits = crédits.

\sphinxAtStartPar
Par exemple, une écriture d’achat se schématise ainsi :
\begin{itemize}
\item {} 
\sphinxAtStartPar
une ou plusieurs lignes au débit des comptes de charges correspondant aux biens acquis ou aux services reçus (matériel, prime d’assurance…)

\item {} 
\sphinxAtStartPar
une ligne au crédit du compte tiers fournisseur pour le net\sphinxhyphen{}à\sphinxhyphen{}payer de la facture

\end{itemize}

\sphinxAtStartPar
Pour ce qui est des comptes de tiers, vous pouvez aussi sous \sphinxstyleemphasis{Diacamma} lettrer les lignes d’écriture.

\sphinxAtStartPar
Ex: la dette née d’un achat avec son règlement.


\subsection{Plan comptable de l’exercice}
\label{\detokenize{accounting/definition:plan-comptable-de-l-exercice}}
\sphinxAtStartPar
Le plan comptable de l’exercice est l’ensemble des comptes ouverts au titre de l’exercice comptable. Il doit respecter  la nomenclature officielle des comptes applicable dans votre pays.
\begin{description}
\item[{Sous {\color{red}\bfseries{}*}Diacamma », le plan comptable est spécifique à chaque exercice comptable. Vous pourrez le créer :}] \leavevmode\begin{itemize}
\item {} 
\sphinxAtStartPar
soit  à partir du plan de comptes type mis à votre disposition par le logiciel

\item {} 
\sphinxAtStartPar
soit en recopiant le plan de comptes de l’exercice précédent

\end{itemize}

\end{description}

\sphinxAtStartPar
La création des comptes n’est pas totalement libre. Vous pouvez subdiviser un compte. Par contre, vous ne devez pas créer un compte hors de la nomenclature officielle.


\section{Exercice comptable}
\label{\detokenize{accounting/fiscalyear:exercice-comptable}}\label{\detokenize{accounting/fiscalyear::doc}}

\subsection{Paramétrages}
\label{\detokenize{accounting/fiscalyear:parametrages}}\begin{quote}

\sphinxAtStartPar
Menu \sphinxstyleemphasis{Administration/Modules (conf.)/Configuration comptable}
\end{quote}

\sphinxAtStartPar
Ouvrez l’onglet « Paramètres » et éditez\sphinxhyphen{}les avec le bouton « Modifier ». Paramétrez la devise et sa précision, la taille des codes comptables. Précisez aussi si vous avez l’intention ou non de mettre en place une comptabilité analytique.
\begin{quote}

\noindent\sphinxincludegraphics{{parameters}.png}
\end{quote}


\subsection{Création d’un exercice comptable}
\label{\detokenize{accounting/fiscalyear:creation-d-un-exercice-comptable}}
\sphinxAtStartPar
Lors de la mise en place de votre comptabilité sous \sphinxstyleemphasis{Diacamma Syndic}, vous aurez à spécifier le système comptable qui sera utilisé (ex. Plan comptable général français). \sphinxstylestrong{Attention :} une fois choisie, cette option ne sera plus modifiable.
\begin{quote}
\begin{quote}

\sphinxAtStartPar
Menu \sphinxstyleemphasis{Administration/Modules (conf.)/Configuration comptable} \sphinxhyphen{} Onglet « Liste d’exercices »
\end{quote}

\noindent\sphinxincludegraphics{{fiscalyear_list}.png}
\end{quote}

\sphinxAtStartPar
Vérifiez que le système comptable a été choisi et cliquez sur « + Créer » afin de renseigner les bornes du nouvel exercice.
\begin{quote}

\noindent\sphinxincludegraphics{{fiscalyear_create}.png}
\end{quote}

\sphinxAtStartPar
Pour le premier exercice sous \sphinxstyleemphasis{Diacamma Syndic}, saisissez la date de début et la date de fin de l’exercice puis cliquez sur le bouton « OK ». Les exercices suivants ont comme date de début le lendemain de la date de clôture de l’exercice précédent et seule la date de fin est à saisir.

\sphinxAtStartPar
Notez que le logiciel associe à chaque exercice un répertoire de stockage du \sphinxstyleemphasis{Gestionnaire de documents} : certains documents
officiels seront sauvegardés dans celui\sphinxhyphen{}ci. Le bouton « Contrôle » vous permet à tout moment de  vérifier que vos documents officiels ont bien été générés.

\sphinxAtStartPar
Votre nouvel exercice figure maintenant dans la liste des exercices. Il est \sphinxstylestrong{{[}en création{]}}. Lorsque plusieurs exercices ont été créés, vous devez activer celui sur lequel vous souhaitez travailler par défaut, à l’aide du bouton « Activé ».
\begin{quote}

\sphinxAtStartPar
\sphinxstyleemphasis{Onglet « Journaux » et Onglet « Champ personnalisé des tiers »}
\end{quote}

\sphinxAtStartPar
Depuis ce même écran de configuration, vous pouvez également modifier ou ajouter des journaux. Des champs personnalisés peuvent aussi être ajoutés à la fiche modèle de tiers comptable. Ceci peut être intéressant si vous voulez réaliser des recherches/filtrages sur des informations propres à votre fonctionnement.

\sphinxAtStartPar
Maintenant, vous devez fermer la fenêtre « Configuration comptable » et créer le plan comptable de votre structure.
\begin{quote}
\begin{quote}

\sphinxAtStartPar
Menu \sphinxstyleemphasis{Comptabilité/Gestion comptable/Plan comptable}
\end{quote}
\end{quote}

\sphinxAtStartPar
Avec le bouton « + Initiaux », générez automatiquement votre propre plan comptable général à partir du plan de comptes type fourni par le logiciel.
Adaptez celui\sphinxhyphen{}ci aux besoins de votre structure avec les boutons « Ajouter » et « Supprimer ».
\begin{itemize}
\item {} \begin{description}
\item[{\sphinxstylestrong{Première tenue de comptabilité sous Diacamma}}] \leavevmode{[}vous migrez sous Diacamma et avez des à\sphinxhyphen{}nouveaux à saisir.{]}
\sphinxAtStartPar
Avec le bouton « + Initiaux », générez automatiquement votre propre plan comptable général à partir du plan de comptes type.
Adaptez celui\sphinxhyphen{}ci aux besoins de votre structure avec les boutons « Ajouter » et « Supprimer ».
Quittez l’écran \sphinxstyleemphasis{Plan comptable} et ouvrez le menu \sphinxstyleemphasis{Comptabilité/Gestion comptable/Ecritures comptables}.
Saisissez vos soldes à\sphinxhyphen{}nouveaux en une seule écriture, en prenant bien soin de la contrôler, dans le journal « Report à nouveau ».
Ceci fait, réouvrez le menu \sphinxstyleemphasis{Comptabilité/Gestion comptable/Plan comptable}.

\end{description}

\item {} \begin{description}
\item[{\sphinxstylestrong{Exercice comptable suivant}}] \leavevmode{[}Il ne s’agit pas de votre premier exercice sous \sphinxstyleemphasis{Diacamma}.{]}
\sphinxAtStartPar
Si ce n’est pas déjà réalisé, avec le bouton « Importer » (et non « Initiaux »), vous devez importer le plan comptable de l’exercice précédent.
Contrôlez l’importation et mettez à jour, si besoin, le plan comptable de l’exercice.
Suite à votre dernière assemblée générale, les excédents n\sphinxhyphen{}1 doivent être normalement ventilés avant clôture de l’exercice n\sphinxhyphen{}1. Si ce n’est pas le cas, vous devez passer cette écriture.
Clôturez l’exercice n\sphinxhyphen{}1. Son état est maintenant \sphinxstylestrong{{[}terminé{]}}. Le nouvel exercice est toujours \sphinxstylestrong{{[}en création{]}}.
Utilisez le bouton « Report à nouveau » afin que les soldes n\sphinxhyphen{}1 des comptes d’actif et de passif soient repris dans la comptabilité du nouvel exercice. Vous pouvez constater à l’écran que les soldes des comptes de bilan non soldés fin n\sphinxhyphen{}1 ont été repris. L’écriture correspondante  au journal « Report à nouveau » est générée et validée automatiquement.

\end{description}

\end{itemize}

\sphinxAtStartPar
\sphinxstylestrong{Afin d’achever l’ouverture de votre nouvel exercice, vous devez maintenant cliquez sur « Commencer ».}
\sphinxstylestrong{Votre exercice est maintenant {[}en cours{]}}.


\subsection{Création, modification et édition de comptes du plan comptable}
\label{\detokenize{accounting/fiscalyear:creation-modification-et-edition-de-comptes-du-plan-comptable}}\begin{quote}

\sphinxAtStartPar
Menu \sphinxstyleemphasis{Comptabilité/Gestion comptable/Plan comptable}
\end{quote}

\sphinxAtStartPar
A tout moment vous pouvez ajouter un nouveau compte dans votre plan comptable.
\begin{quote}

\noindent\sphinxincludegraphics{{account_new}.png}
\end{quote}

\sphinxAtStartPar
Référez\sphinxhyphen{}vous à la règlementation de votre pays. Votre structure peut avoir l’obligation de respecter certaines obligations pour ce qui est de leur plan comptable.

\sphinxAtStartPar
Un compte peut être modifié, tant pour ce qui est de son numéro que de son intitulé. Les imputations (lignes d’écritures) qui lui sont associées seront automatiquement modifiées. Le changement n’est permis que si le nouveau compte relève de la même catégorie comptable (charge, produit…).

\sphinxAtStartPar
Lorsque vous consultez un compte (bouton « Editer » ou double\sphinxhyphen{}clic), les écritures associées au compte sont affichées.
\begin{quote}

\noindent\sphinxincludegraphics{{account_edit}.png}
\end{quote}

\sphinxAtStartPar
Il vous est aussi possible de supprimer un compte du plan comptable à la condition qu’aucune écriture ne lui soit associée.


\subsection{Clôture d’un exercice}
\label{\detokenize{accounting/fiscalyear:cloture-d-un-exercice}}
\sphinxAtStartPar
En fin d’exercice comptable, celui\sphinxhyphen{}ci est clôturé. Cette opération, définitive, se réalise sous le contrôle de votre
vérificateur aux comptes.

\sphinxAtStartPar
Au préalable, vous devez :
\begin{itemize}
\item {} 
\sphinxAtStartPar
Passer vos écritures d’inventaire (charges à payer, produits à recevoir, créances douteuses…)

\item {} 
\sphinxAtStartPar
Contrôler que toutes les charges et les produits ont bien été imputés en comptabilité analytique

\item {} 
\sphinxAtStartPar
Vérifier que vos dépenses et vos recettes sont bien ventilées sur vos différentes catégories

\item {} 
\sphinxAtStartPar
Vérifier que toutes vos dépenses ont été ventilées sur les copropriétaires, pour ce qui est des copropriétés

\item {} 
\sphinxAtStartPar
Affecter vos excédents conformément aux délibérations de votre assemblée générale

\item {} 
\sphinxAtStartPar
Valider les écritures provisoires au brouillard

\item {} 
\sphinxAtStartPar
Lettrer les comptes de tiers

\item {} 
\sphinxAtStartPar
Créer l’exercice suivant si cela n’a pas été réalisé

\item {} 
\sphinxAtStartPar
Sauvegarder votre dossier
\begin{quote}

\sphinxAtStartPar
Menu \sphinxstyleemphasis{Comptabilité/Gestion comptable/Plan comptable}
\end{quote}

\end{itemize}

\sphinxAtStartPar
Cliquez sur le bouton « Clôturer ».

\sphinxAtStartPar
La clôture a pour effet de :
\begin{itemize}
\item {} 
\sphinxAtStartPar
Solder les comptes de gestion

\item {} 
\sphinxAtStartPar
Interdire tout ajout d’écriture

\item {} 
\sphinxAtStartPar
Arrêter les comptes de bilan et les comptes de tiers (copropriétaires, fournisseurs…)

\item {} 
\sphinxAtStartPar
Assurer qu’il ne pourra plus être apporté de modification à l’exercice clôturé

\end{itemize}

\sphinxAtStartPar
\sphinxstylestrong{Remarques :}
\begin{itemize}
\item {} 
\sphinxAtStartPar
Tant qu’un exercice n’est pas clôturé, vous pouvez enregistrer des opérations sur celui\sphinxhyphen{}ci et le suivant

\item {} 
\sphinxAtStartPar
Certaines structures ont des règles de clôture spécifique (exemple les ASL): bien verifier votre règlementation comptable en la matière.

\end{itemize}


\section{Tiers comptable}
\label{\detokenize{accounting/third:tiers-comptable}}\label{\detokenize{accounting/third::doc}}
\sphinxAtStartPar
\sphinxstylestrong{Préambule} :
Les opérations réalisées par votre structure avec ses partenaires donnent lieu à des échanges. Les flux physiques (échanges de biens et de services) et les flux monétaires (échanges de monnaie) doivent être constatés en comptabilité.

\sphinxAtStartPar
Pour chacun de vos partenaires, que ce soit des personnes physiques (individus) ou des personnes morales (administrations, associations, sociétés…) vous devez, afin de comptabiliser distinctement les opérations réalisées avec lui, lui ouvrir une subdivision d’un compte général appelé aussi compte auxiliaire (ou compte nominatif).

\sphinxAtStartPar
Exemple : vous achetez régulièrement des fournitures de bureau à la SARL « Papeterie du Centre » dont la gérante est Mme VERGER Louise
\begin{description}
\item[{Sous DIACAMMA :}] \leavevmode\begin{description}
\item[{Fonction}] \leavevmode{[}Gérante                                  \textendash{}\textgreater{}{]}\begin{description}
\item[{Personne physique}] \leavevmode{[}VERGER Louise{]}\begin{description}
\item[{Personne morale}] \leavevmode{[}Papeterie du Centre (SARL) avec pour membre VERGER Louise, gérante{]}
\sphinxAtStartPar
Tiers comptable : Papeterie du Centre \sphinxhyphen{} Code 401

\end{description}

\end{description}

\end{description}

\end{description}


\subsection{Création d’un tiers comptable}
\label{\detokenize{accounting/third:creation-d-un-tiers-comptable}}\begin{quote}

\sphinxAtStartPar
Menu \sphinxstyleemphasis{Comptabilité/Tiers}
\end{quote}

\noindent\sphinxincludegraphics{{third_list}.png}

\sphinxAtStartPar
La liste des tiers précédemment enregistrés s’affiche à l’écran. Dans cette liste, notez la présence des membres de votre structure.
Vous pouvez filtrer cette liste par contact ou type de tiers et imprimer la liste obtenue.

\sphinxAtStartPar
Pour ajouter un nouveau tiers comptable, cliquez sur « + Créer » et sélectionnez le contact (personne physique ou morale) associé à ce tiers comptable.

\noindent\sphinxincludegraphics{{third_add}.png}

\sphinxAtStartPar
Depuis cet écran, vous pouvez aussi créer un nouveau contact à l’aide du bouton « + Créer » avant de le sélectionner.

\noindent\sphinxincludegraphics{{third_edit}.png}

\sphinxAtStartPar
Il est possible d’associer un tiers comptable à un ou plusieurs comptes comptables : fournisseur, client…. Cela permet en comptabilité de distinguer les opérations selon la nature des échanges que vous avez avec lui.


\subsection{Situation d’un tiers}
\label{\detokenize{accounting/third:situation-d-un-tiers}}
\sphinxAtStartPar
L’onglet « Ecritures comptables » de la fiche d’un tiers vous donne une vue sur les écritures comptables le concernant,
avec possibilité de les filtrer.

\noindent\sphinxincludegraphics{{third_state}.png}


\subsection{Champs personnalisés de tiers}
\label{\detokenize{accounting/third:champs-personnalises-de-tiers}}\begin{quote}

\sphinxAtStartPar
Menu \sphinxstyleemphasis{Administration/Modules (conf.)/Configuration comptable} \sphinxhyphen{} onglet « Champ personnalisé de tiers »
\end{quote}

\sphinxAtStartPar
Grâce à l’ajout de vos propres champs dans la fiche type de tiers comptable, vous allez pouvoir personnaliser cette fiche.
La méthode est similaire à celle utilisée pour personnaliser la fiche des contacts.


\section{Écritures}
\label{\detokenize{accounting/entity:ecritures}}\label{\detokenize{accounting/entity::doc}}
\sphinxAtStartPar
Avant toute saisie d’écriture, assurez\sphinxhyphen{}vous de l’existence de vos journaux :
\begin{quote}

\sphinxAtStartPar
Menu \sphinxstyleemphasis{Administration/Modules(conf.)/Configuration comptable} onglet « Journaux »
\end{quote}

\sphinxAtStartPar
De même, vous devez contrôler que votre plan comptable est à jour.
\begin{quote}

\sphinxAtStartPar
Menu \sphinxstyleemphasis{Comptabilité/Gestion comptable/Plan comptable}
\end{quote}


\subsection{Saisie d’une écriture}
\label{\detokenize{accounting/entity:saisie-d-une-ecriture}}

\subsubsection{Cas général}
\label{\detokenize{accounting/entity:cas-general}}\begin{quote}
\begin{quote}

\sphinxAtStartPar
Menu \sphinxstyleemphasis{Comptabilité/Gestion comptable/Écritures comptables}
\end{quote}

\noindent\sphinxincludegraphics{{entity_list}.png}
\end{quote}

\sphinxAtStartPar
Depuis cet écran, vous avez la possibilité de visualiser les écritures précédemment saisies et vous pouvez en ajouter de nouvelles.
A l’écran, vous pouvez aussi consulter les écritures saisies après les avoir filtrées sur l’exercice comptable, un journal (ou tous)  et/ou sur l’état des écritures :
\begin{itemize}
\item {} 
\sphinxAtStartPar
Tout : aucun filtrage n’est appliqué

\item {} 
\sphinxAtStartPar
En cours (brouillard): seulement les écritures provisoires (non encore validées)

\item {} 
\sphinxAtStartPar
Validée : seulement les écritures déjà validées

\item {} 
\sphinxAtStartPar
Lettrée : seulement les écritures lettrées

\item {} 
\sphinxAtStartPar
Non lettrée : seulement les écritures non encore lettrées

\end{itemize}

\sphinxAtStartPar
Vous pouvez également indiquer un code comptable pour filtre les écritures associées.

\sphinxAtStartPar
Pour saisir une nouvelle écriture, cliquez sur le bouton « + Ajouter ».
Sélectionnez le journal et saisissez la date de l’opération ainsi que le libellé de l’écriture (pièce et numéro…). Cliquez sur le bouton « Modifier ».

\sphinxAtStartPar
Pour chaque ligne de l’écriture :
\begin{itemize}
\item {} 
\sphinxAtStartPar
Saisissez les premiers chiffres du numéro du compte devant être mouvementé et sélectionnez\sphinxhyphen{}le dans votre plan comptable. Un compte doit exister dans le plan comptable de l’exercice pour être mouvementable

\item {} 
\sphinxAtStartPar
Si demandé, sélectionnez le tiers et le code analytique de rattachement

\item {} 
\sphinxAtStartPar
Saisissez le montant au débit ou au crédit du compte spécifié

\item {} 
\sphinxAtStartPar
Cliquez sur le bouton « + Ajouter ».

\end{itemize}

\sphinxAtStartPar
Ceci fait, cliquez sur « Ok ». Vous ne pourrez valider la saisie de votre écriture que si elle est équilibrée (débits = crédits).

\sphinxAtStartPar
\sphinxstylestrong{Après clôture de l’exercice comptable, il ne sera plus possible de passer une écriture sur l’exercice clos. Avant cela, prenez soin de passer vos dernières écritures.}

\sphinxAtStartPar
Une écriture étant provisoire, le bouton « Inverser » vous permet d’inverser très facilement votre écriture si besoin.
Quand on débute en comptabilité, on a parfois du mal à savoir si un compte doit être débité ou crédité.
En saisie, lorsque vous débitez un compte fournisseur et créditez un compte de charge, un message vous alerte sur le fait que vous avez probablement saisi l »écriture d’un avoir ».


\subsubsection{Comptabiliser un règlement}
\label{\detokenize{accounting/entity:comptabiliser-un-reglement}}
\sphinxAtStartPar
Un règlement peut être saisi manuellement comme précédemment. Mais bien souvent il est lié à une opération déjà comptabilisée  (ex. achat, appel de fonds) et constatant une dette ou une créance.

\sphinxAtStartPar
Pour simplifier votre saisie, éditez l’écriture constatant la dette ou la créance réglée. Cliquez sur le bouton « Règlement » : l’application vous propose alors une nouvelle écriture partiellement remplie avec le compte du tiers débité (règlement de dette) ou crédité (règlement de créance).
Il ne vous reste plus qu’à préciser sur quel compte financier (caisse, banque…) vous voulez imputer le règlement et à contrôler l’écriture générée après l’avoir complétée.

\sphinxAtStartPar
Une fois l’écriture de règlement validée via cette fonctionnalité, l’écriture d’origine de la dette ou de la créance et celle du règlement sont automatiquement associées. Les lignes concernant le compte de tiers seront lettrées automatiquement.


\subsubsection{Écriture d’à\sphinxhyphen{}nouveaux}
\label{\detokenize{accounting/entity:ecriture-d-a-nouveaux}}
\sphinxAtStartPar
Après clôture d’un exercice comptable, l’écriture d’à\sphinxhyphen{}nouveaux est automatiquement générée lors de la phase d’initialisation de l’exercice suivant. Cette écriture, passée dans le journal « Report à nouveau », est automatiquement validée.
A ce moment, vous pouvez être amené à saisir des opérations spécifiques comme par exemple la ventilation des excédents de l’année précédente.

\sphinxAtStartPar
Par contre, dans ce journal, vous ne pouvez pas enregistrer de charges ou de produits.


\subsection{Lettrage d’écritures}
\label{\detokenize{accounting/entity:lettrage-d-ecritures}}
\sphinxAtStartPar
Comme nous l’avons évoqué dans un précédent chapitre, il est fréquent que des mouvements enregistrés en comptabilité trouvent leur source dans une ou plusieurs opérations liées. Dans ce cas, les lignes d’écritures correspondantes peuvent être lettrées.

\sphinxAtStartPar
Toutefois, il y a des conditions pour que des lignes d’écriture soient lettrables :
\begin{itemize}
\item {} 
\sphinxAtStartPar
Elles doivent concerner le même compte de tiers (fournisseur ou copropriétaire)

\item {} 
\sphinxAtStartPar
Pour être lettrables, le total des débits doit être égal au total des crédits

\item {} 
\sphinxAtStartPar
Soit les lignes sont d’un même exercice non clôturé (il faut donc lettrer les comptes de tiers avant clôture de l’exercice)

\item {} 
\sphinxAtStartPar
Soit les lignes sont sur deux exercices non clôturés qui ce suivent. Le lettrage sera alors temporaire (avec un « \& » en suffix) et corrigé définitivement au moment de la clôture et du report à\sphinxhyphen{}nouveau.

\end{itemize}

\sphinxAtStartPar
Les lignes lettrées conjointement se voient attribuer le même code lettre.

\sphinxAtStartPar
Toute ligne lettrée peut être délettrée.


\subsection{Validation d’écritures}
\label{\detokenize{accounting/entity:validation-d-ecritures}}
\sphinxAtStartPar
Par défaut, une écriture est saisie au brouillard, ce qui permet de la modifier ou de la supprimer tant qu’elle n’est pas validée.
Cette écriture doit être validée pour entériner votre saisie. En principe, cette validation est confiée à la personne en charge de la vérification de la comptabilisation des opérations.

\sphinxAtStartPar
Pour réaliser cette action, sélectionnez les écritures contrôlées et cliquez sur le bouton « Clôturer »: L’application affectera alors un numéro aux écritures validées ainsi qu’une date de validation.

\sphinxAtStartPar
Une fois validée, une écriture devient non modifiable : ce mécanisme assure le caractére intangible et irréversible de votre comptabilité.

\sphinxAtStartPar
Comme l’erreur est humaine, l’écriture validée ne pouvant pas être modifiée ou supprimée, vous devrez procéder comme suit :
\begin{itemize}
\item {} 
\sphinxAtStartPar
1 : Contrepasser l’écriture erronée en créant une écriture inverse pour l’annuler. Le libellé doit spécifier la référence de l’écriture annulée avec la mention « Contrepassation… »

\item {} 
\sphinxAtStartPar
2 : Enregistrer l’écriture correcte

\end{itemize}

\sphinxAtStartPar
\sphinxstylestrong{Avant clôture de l’exercice comptable, toutes les écritures doivent étre validées}.


\subsection{Recherche d’écriture(s)}
\label{\detokenize{accounting/entity:recherche-d-ecriture-s}}
\sphinxAtStartPar
Depuis la liste des écritures, le bouton « Recherche » vous permet de définir les critères de recherche d’écritures comptables.
\begin{quote}

\noindent\sphinxincludegraphics{{entity_search}.png}
\end{quote}

\sphinxAtStartPar
En cliquant sur « Recherche », l’outil va rechercher dans la base toutes les écritures satisfaisant aux critères saisis.
La ou les écritures extraites pourront être :
\begin{itemize}
\item {} 
\sphinxAtStartPar
Imprimées

\item {} 
\sphinxAtStartPar
Éditées/modifiées

\item {} 
\sphinxAtStartPar
Clôturée, lettrées ou délettrées…

\end{itemize}


\subsection{Import d’écritures}
\label{\detokenize{accounting/entity:import-d-ecritures}}
\sphinxAtStartPar
Depuis la liste des écritures, le bouton « Import » vous permet d’importer des écritures comptables depuis un fichier CSV.

\sphinxAtStartPar
Après avoir sélectionné l’exercice d’import, le journal et les informations de format de votre fichier CSV, vous devez associer les champs des écritures aux colonnes de votre document (la première ligne de votre document doit décrire la nature de chaque colonne).
\begin{quote}

\noindent\sphinxincludegraphics{{entity_import}.png}
\end{quote}

\sphinxAtStartPar
Vous pouvez alors contrôler vos données avant de les valider.
L’import réalisé, l’outil vous présentera le résultat des écritures réellement importées.

\sphinxAtStartPar
\sphinxstylestrong{Notez que les lignes d’écritures ne seront pas importées si :}
\begin{itemize}
\item {} 
\sphinxAtStartPar
Le code comptable précisé n’existe pas dans le plan comptable de l’exercice

\item {} 
\sphinxAtStartPar
La date n’est pas inclue dans l’exercice comptable actif

\item {} 
\sphinxAtStartPar
Le principe de la partie double n’est pas respecté car pour toute opération, le total des débits doit être égal au total des crédits

\end{itemize}

\sphinxAtStartPar
Bien que cela ne bloque pas l’import, le tiers et le code analytique seront laissés vides si ceux indiqués ne sont pas référencés dans votre dossier comptable. Vous devez donc contrôler l’importation et la modifier si besoin.


\section{Comptabilité analytique}
\label{\detokenize{accounting/costaccounting:comptabilite-analytique}}\label{\detokenize{accounting/costaccounting::doc}}
\sphinxAtStartPar
Pour réaliser une analyse financière des différentes activités de votre structure et déterminer le résultat de chacune d’elles, vous pouvez mettre en place une comptabilité analytique.
\begin{quote}

\sphinxAtStartPar
Menu \sphinxstyleemphasis{Administration/Modules(conf.)/Configuration comptable}
\end{quote}

\sphinxAtStartPar
Après avoir ouvert l’onglet « Paramètres » \sphinxhyphen{} bouton « Modifier », cochez le paramètre \sphinxstyleemphasis{Comptabilité analytique}. A noter qu’en activant ce paramètre, toutes les charges et les produits devront obligatoirement être repris en comptabilité analytique.

\sphinxAtStartPar
La comptabilité analytique proposée par le logiciel est une version simplifiée. En effet, il n’est pas possible de ventiler une charge ou un produit sur plusieurs codes analytiques.


\subsection{Les codes analytiques}
\label{\detokenize{accounting/costaccounting:les-codes-analytiques}}\begin{quote}

\sphinxAtStartPar
Menu \sphinxstyleemphasis{Comptabilité/Gestion comptable/Comptabilités analytiques}
\end{quote}

\sphinxAtStartPar
Vous accédez à la liste des codes analytiques créés.

\sphinxAtStartPar
Depuis cet écran vous pouvez créer, modifier ou supprimer un code analytique.

\sphinxAtStartPar
Chaque code a un titre, un descriptif et un statut (ouvert ou clôturé).
En mode « liste », figure le résultat comptable (les produits diminués des charges) de chaque code analytique.
\begin{quote}

\noindent\sphinxincludegraphics{{costaccount_list}.png}
\end{quote}

\sphinxAtStartPar
Par défaut, un filtrage vous permet de ne voir que  les codes analytiques ouverts. Vous pouvez paramétrer le filtre à l’aide des boutons de liste.


\subsection{Imputation analytique d’une charge ou d’un produit}
\label{\detokenize{accounting/costaccounting:imputation-analytique-d-une-charge-ou-d-un-produit}}\begin{quote}

\sphinxAtStartPar
Menu \sphinxstyleemphasis{Comptabilité/Gestion comptable/Ecritures comptables}
\end{quote}

\sphinxAtStartPar
Si vous avez des codes analytiques ouverts, vous pouvez imputer une charge ou un produit sur l’un d’entre eux.
\begin{quote}

\noindent\sphinxincludegraphics{{costaccount_assign}.png}
\end{quote}

\sphinxAtStartPar
Que l’écriture correspondante soit validée ou non, affichez cette écriture  et éditez\sphinxhyphen{}la.
Sélectionnez le  mouvement relatif à la charge ou au produit à imputer en analytique et bouton « Modifier ».
Renseignez le code analytique de rattachement.

\sphinxAtStartPar
Il est aussi possible de réaliser cette imputation par lot.
Sélectionnez les mouvements à affecter et cliquez sur le bouton \sphinxstyleemphasis{Analytique}. Choisissez alors le nouveau code à utiliser
pour l’ensemble des mouvements (charges ou  produits) sélectionnés.


\subsection{Impressions analytiques}
\label{\detokenize{accounting/costaccounting:impressions-analytiques}}
\sphinxAtStartPar
Depuis la liste des codes analytiques, vous pouvez réaliser un rapport type « Compte de résultat de comptabilité analytique ».
Pour cela, affichez les codes analytiques. Sélectionnez\sphinxhyphen{}les (un ou plusieurs) et cliquez sur le bouton « Rapport ».
Tout comme le Compte de résultat de la Comptabilité générale, ce rapport reprend pour chaque code analytique sélectionné, le montant des charges et des produits ainsi que le budget. Cela permet de comparer, là encore le résultat réel d’une activité au résultat attendu.


\section{Modèle d’écriture}
\label{\detokenize{accounting/model:modele-d-ecriture}}\label{\detokenize{accounting/model::doc}}

\subsection{Déclaration d’un modèle}
\label{\detokenize{accounting/model:declaration-d-un-modele}}
\sphinxAtStartPar
Certaines opérations sont régulièrement enregistrées. Pour soulager la saisie de celles\sphinxhyphen{}ci, un modèle d’écriture ou masque de saisie peut être créé et enregistré afin d’être utilisé ultérieurement en saisie.
Un modèle d’écriture peut aussi être utilisé pour faciliter la comptabilisation d’opérations complexes.

\sphinxAtStartPar
Le modèle se présente comme une écriture.
\begin{quote}
\begin{quote}

\sphinxAtStartPar
Menu \sphinxstyleemphasis{Comptabilité/Gestion comptable/Modèles d’écritures}
\end{quote}

\noindent\sphinxincludegraphics{{model_list}.png}
\end{quote}

\sphinxAtStartPar
Il est associé à un journal, contient un descriptif, précise les comptes à débiter et ceux à créditer, avec indication des montants qui ne peuvent pas être laissés nuls.
\begin{quote}

\noindent\sphinxincludegraphics{{model_item}.png}
\end{quote}

\sphinxAtStartPar
Nous vous conseillons de créer un modèle pour chacune de vos dépenses ou recettes règulières. Ainsi, vous gagnerez du temps sur la saisie de votre comptabilité et n’aurez pas à rechercher les bons codes comptables ni le sens des imputations.


\subsection{Utilisation d’un modèle}
\label{\detokenize{accounting/model:utilisation-d-un-modele}}
\sphinxAtStartPar
L’utilisation d’un modèle est très simple.

\sphinxAtStartPar
En saisie d’écriture, cliquez sur le bouton « + Modèle ».
\begin{quote}

\noindent\sphinxincludegraphics{{model_add}.png}
\end{quote}

\sphinxAtStartPar
Sélectionnez votre modèle et précisez le coefficient multiplicateur qui devra être appliqué au montant présaisi dans le modèle. Ce facteur est très pratique lorsque l’on a des factures récurrentes mais dont le montant peut fluctuer. Il est alors possible, è l’aide de ce réel, de le faire varier.
Validez votre sélection par « Ok ». Une écriture est générée d’après le modèle. Vous pouvez la corriger comme n’importe quelle écriture.


\section{Budgets prévisionnels}
\label{\detokenize{accounting/budget:budgets-previsionnels}}\label{\detokenize{accounting/budget::doc}}

\subsection{Budget analytique}
\label{\detokenize{accounting/budget:budget-analytique}}\begin{quote}

\sphinxAtStartPar
Menu \sphinxstyleemphasis{Comptabilité/Gestion comptable/Comptabilités analytiques}
\end{quote}

\sphinxAtStartPar
Depuis l’interface des comptabilités analytiques, vous pouvez ajouter un budget prévisionnel à chacun.

\sphinxAtStartPar
Cliquez simplement sur le bouton « Budget » après avoir sélectionné la comptabilité à compléter.

\sphinxAtStartPar
L’interface vous permet alors d’ajouter (comptes et montants) des charges ou des produits ; le budget peut être en déséquilibre.
Vous pouvez également importer les montants des charges et produits réels d’une comptabilité précédente.

\sphinxAtStartPar
Ce budget prévisionnel apparait alors dans les rapports afin de le comparer avec le réalisé.


\subsection{Budget prévisionnel général}
\label{\detokenize{accounting/budget:budget-previsionnel-general}}\begin{quote}

\sphinxAtStartPar
Menu \sphinxstyleemphasis{Comptabilité/Gestion comptable/Plan comptable}
\end{quote}

\sphinxAtStartPar
Depuis l’interface du plan comptable courant, vous pouvez renseigner le budget prévisionnel de l’exercice via le bouton « Budget ».

\sphinxAtStartPar
Comme pour un budget analytique, vous pouvez ajouter des comptes de charges ou de produits ainsi qu’importer le réalisé d’un exercice précédent.
A noter qu’automatiquement, l’ensemble des budgets analytiques associés au même exercice sont automatiquement consolidés dans le budget prévisionnel général.

\sphinxAtStartPar
Le \sphinxstyleemphasis{Compte de résultat} reprend également le budget prévisionnel à des fins de comparaison.


\section{Rapports}
\label{\detokenize{accounting/reporting:rapports}}\label{\detokenize{accounting/reporting::doc}}\begin{quote}

\sphinxAtStartPar
Menu \sphinxstyleemphasis{Comptabilité/Rapports comptables}
\end{quote}

\sphinxAtStartPar
Dans cette catégorie, vous accédez à l’ensemble des documents de synthèse élaborés en fin d’exercice. Les rapports sont obtenus à partir de tous les enregistrements comptables passés pendant l’exercice comptable.

\sphinxAtStartPar
En cours d’exercice, tout rapport peut être consulté à l’écran, sauvegardé dans le gestionnaire de documents au format PDF ou imprimé. Ultérieurement, vous pourrez aussi consulter un rapport sauvegardé et en réaliser une impression tel qu’il a été sauvegardé ou en le régénérant sur un autre modèle.

\sphinxAtStartPar
A la clôture de l’exercice, l’ensemble de ces rapports sont générés et sauvegardés automatiquement dans le gestionnaire de documents. Lorsque vous les éditerez, par défaut vous téléchargerez la sauvegarde. Vous pourrez régénérer un nouveau PDF sur un autre modèle. Par contre celui\sphinxhyphen{}ci comportera la mention « duplicata » en filigrane.

\sphinxAtStartPar
Dans le bilan comptable et le compte de résultat, le budget de l’exercice est reporté afin de permettre la comparaison réalisé\sphinxhyphen{}prévisionnel.


\subsection{Compte de résultat}
\label{\detokenize{accounting/reporting:compte-de-resultat}}
\sphinxAtStartPar
Le compte de résultat  synthétise l’ensemble des charges et des produits de l’exercice comptable.
Il met en évidence le résultat net, c’est\sphinxhyphen{}à\sphinxhyphen{}dire la différence entre vos produits et vos charges. Il y a excédent quand les produits excèdent les charges et inversement, un déficit.


\subsection{Bilan comptable}
\label{\detokenize{accounting/reporting:bilan-comptable}}
\sphinxAtStartPar
Le bilan comptable est une photographie (c’est un instantané) du patrimoine de l’entreprise qui permet l’évaluation d’une structure, et plus précisément de savoir après retraitement (par exemple d’une optique patrimoniale à celle sur option de juste valeur pour l’adoption des normes internationales) combien elle vaut et si elle est solvable.
Au bilan, le résultat qui y figure est égal au solde du compte de résultat (excédent ou déficit)

\sphinxAtStartPar
Il existe trois finalités au bilan comptable :
\begin{itemize}
\item {} 
\sphinxAtStartPar
Le bilan interne, généralement détaillé, utilisé par les responsables de la structure pour différentes analyses internes

\item {} 
\sphinxAtStartPar
Le bilan officiel, destiné aux contrôleurs de la comptabilité (auditeurs et commissaires aux comptes) et aux tiers (actionnaires, banques, clients, salariés, collectivités…).

\item {} 
\sphinxAtStartPar
Le bilan fiscal qui sert à déterminer le bénéfice imposable

\end{itemize}


\subsection{Grand livre}
\label{\detokenize{accounting/reporting:grand-livre}}
\sphinxAtStartPar
Le grand livre est le recueil de l’ensemble des comptes utilisés par une structure dans le cadre de la tenue de sa comptabilité.
Il faut distinguer le grand livre général (comptes des classes 1 à 7) des grands livres auxiliaires avec le détail des comptes de tiers (clients, fournisseurs, associés, copropriétaires).

\sphinxAtStartPar
vous pouvez paramétrer vos éditions afin de les personnaliser, avec :
\begin{itemize}
\item {} 
\sphinxAtStartPar
La période (dates de début et de fin)

\item {} 
\sphinxAtStartPar
Le code comptable commençant par

\item {} 
\sphinxAtStartPar
Ecritures non\sphinxhyphen{}lettrées ou toutes

\end{itemize}


\subsection{Balance}
\label{\detokenize{accounting/reporting:balance}}
\sphinxAtStartPar
Elle est établie pour une période donnée, à partir de tous les comptes du grand livre de la struture. Tout comme le grand livre, elle aussi peut être générale (comptes des classes 1 à 7) ou auxilaire (comptes de tiers). Elle regroupe toutes les montants enregistrés au débit et au crédit de ces comptes et par différence tous les soldes débiteurs et créditeurs.

\sphinxAtStartPar
La balance générale doit être équilibrée, avec « total des débits » = « total des crédits » et « total des soldes débiteurs » = « total des soldes créditeurs ». Cet équilibre permet de vérifier que le principe de la partie double a bien été respecté lors de la comptabilisation des opérations et que tous les mouvements passés au journal ont bien été reportés dans le grand libre.

\sphinxAtStartPar
Vous pouvez personnaliser vos éditions avec :
\begin{itemize}
\item {} 
\sphinxAtStartPar
L’exercice comptable

\item {} 
\sphinxAtStartPar
La période (dates de début et de fin)

\item {} 
\sphinxAtStartPar
Le code comptable commençant par

\item {} 
\sphinxAtStartPar
Seulement les (comptes) non soldés

\end{itemize}

\sphinxAtStartPar
Vous pouvez aussi spécifier si le détail par tiers est souhaité, ce qui permet de transformer votre balance générale en balance mixte générale\sphinxhyphen{}auxiliaire.


\subsection{Listing des écritures}
\label{\detokenize{accounting/reporting:listing-des-ecritures}}
\sphinxAtStartPar
Aux rapports comptables, s’ajoute l’édition des journaux :
\begin{quote}

\sphinxAtStartPar
Menu \sphinxstyleemphasis{Comptabilité/Gestion comptable/Ecritures comptables}
\end{quote}

\sphinxAtStartPar
Depuis l’écran de la liste des écritures comptables, vous avez la possibilité de les visualiser et de les exporter en PDF ou  au format CSV (ce qui permet l’import dans un tableur).


\subsection{Listing du plan comptable de l’exercice}
\label{\detokenize{accounting/reporting:listing-du-plan-comptable-de-l-exercice}}
\sphinxAtStartPar
Depuis l’écran du plan comptable de l’exercice :
\begin{quote}

\sphinxAtStartPar
Menu \sphinxstyleemphasis{Comptabilité/Gestion comptable/Plan comptable}
\end{quote}

\sphinxAtStartPar
Pour un exercice donné et par type de comptes (ou tous), vous pouvez visualiser, pour l’ensemble des comptes ouverts, le récapitulatif des soldes de début et de fin d’exercice avec l’indication des soldes de fin compte\sphinxhyphen{}tenu des seules écritures validées.

\sphinxAtStartPar
Ce récapitulatif peut être imprimé, exporté au format PDF ou CSV (ce qui permet l’import de vos soldes dans un tableur).


\chapter{Diacamma règlement}
\label{\detokenize{payoff/index:diacamma-reglement}}\label{\detokenize{payoff/index::doc}}
\sphinxAtStartPar
Aide relative aux fonctionnalités de gestion des payements.


\section{Règlement}
\label{\detokenize{payoff/payoff:reglement}}\label{\detokenize{payoff/payoff::doc}}
\sphinxAtStartPar
Depuis un module tel que la facturation (\sphinxstyleemphasis{Menu Facturier/Facture}) ou la gestion de copropriétaire (\sphinxstyleemphasis{Menu Copropriété/Les propriétaires et les lots}, onglet \sphinxstyleemphasis{Appels et règlements}), il vous est possible de gérer des règlement.

\sphinxAtStartPar
Depuis la fiche du document, cliquez sur «ajouter» un règlement.
\begin{quote}

\noindent\sphinxincludegraphics{{payoff1}.png}
\end{quote}

\sphinxAtStartPar
Dans la boîte de dialogue, saisissez le détail du règlement reçu du tiers en précisant le mode de paiement.
7 modes de paiement vous sont proposés:
\begin{itemize}
\item {} 
\sphinxAtStartPar
espèces

\item {} 
\sphinxAtStartPar
chèque

\item {} 
\sphinxAtStartPar
virement

\item {} 
\sphinxAtStartPar
prélèvement

\item {} 
\sphinxAtStartPar
carte de crédit

\item {} 
\sphinxAtStartPar
autre

\item {} 
\sphinxAtStartPar
interne

\end{itemize}

\sphinxAtStartPar
Le mode \sphinxstyleemphasis{espèces} sera lié à un mouvement comptable avec le code référencé dans \sphinxstyleemphasis{Administration/Configuration des règlements} (onglet \sphinxstyleemphasis{Paramètres}, champ « compte de caisse »).
Pour les modes \sphinxstyleemphasis{chèque}, \sphinxstyleemphasis{virement}, \sphinxstyleemphasis{prélèvement}, \sphinxstyleemphasis{carte de crédit} et \sphinxstyleemphasis{autre}, vous devez préciser le compte bancaire sur lequel imputer ce mouvement financier.
Le mode \sphinxstyleemphasis{interne} permet d’utiliser un autre justificatif financier (comme par exemple un avoir) pour « solder » celui\sphinxhyphen{}ci. Un rapporchement ou une écriture de compensation sera alors réalisé en comptabilité.

\sphinxAtStartPar
Vous pourrez voir alors dans les documents liés le net\sphinxhyphen{}à\sphinxhyphen{}payer et le ou les règlements correspondants ainsi que le reste dû.

\sphinxAtStartPar
Chaque règlement génère automatiquement une écriture comptable au journal.

\sphinxAtStartPar
Il est également possible de réaliser un règlement multiple.
Suivant le type de document sur lequel ce paiement est associé, vous pouvez avoir plusieurs modes de répartition:
\begin{itemize}
\item {} 
\sphinxAtStartPar
Par date : le paiement est d’abord ventilé sur la facture la plus ancienne, puis la suivante, etc.

\item {} 
\sphinxAtStartPar
Par prorata : le paiement est automatiquement ventilé sur toutes les factures sélectionnées, au prorata de leurs montants.
\begin{quote}

\noindent\sphinxincludegraphics{{multi-payoff}.png}
\end{quote}

\end{itemize}

\sphinxAtStartPar
Quelque soit le mode de répartition, une seule écriture comptable d’encaissement sera alors générées.


\section{Dépôts de chèques}
\label{\detokenize{payoff/deposit:depots-de-cheques}}\label{\detokenize{payoff/deposit::doc}}\begin{quote}

\sphinxAtStartPar
Menu \sphinxstyleemphasis{Comptabilité/Dépôt de chèques}
\end{quote}

\sphinxAtStartPar
Grâce à ce menu, vous pouvez préparer vos remises de chèques à l’encaissement.
A l’écran s’affiche la liste des bordereaux de dépôt préparés que vous pouvez filtrer sur le statut.
\begin{quote}

\noindent\sphinxincludegraphics{{depositlist}.png}
\end{quote}

\sphinxAtStartPar
Pour faciliter la recherche d’écritures, vous pouvez les filtrer.


\subsection{Création}
\label{\detokenize{payoff/deposit:creation}}
\sphinxAtStartPar
Si vous souhaitez préparer une nouvelle remise, cliquez sur « + Créer ».
Dans la boîte de dialogue ouverte, sélectionnez le code bancaire concerné, la référence du dépôt et la date de la remise.

\sphinxAtStartPar
Le détail du dépôt de chèques est complété avec le bouton « + Ajouter ».
Sélectionnez ensuite dans la liste des créances, celles réglées et cliquez sur le bouton « Sélectionner » afin que les chèques correspondants soient repris dans le détail du bordereau.
\begin{quote}

\noindent\sphinxincludegraphics{{deposititem}.png}
\end{quote}

\sphinxAtStartPar
Votre détail étant maintenant actualisé, vous devez « Clôturer » votre dépôt de chèques après l’avoir contrôlé.


\subsection{Impression}
\label{\detokenize{payoff/deposit:impression}}
\sphinxAtStartPar
Avant de clôturer votre bordereau, vous avez la possibilité de faire une première impression.
Par contre, comme votre bordereau n’est pas finalisé, un filigramm « NON VALIDÉ » apparaitra sur le document.

\sphinxAtStartPar
Une fois cloturé, vous pouvrez imprimer votre dépôt de chèques de façon officiel.
Vous pouvez procéder à la remise effective en banque des chèques assortis de ce document de dépôt.


\subsection{Validation}
\label{\detokenize{payoff/deposit:validation}}
\sphinxAtStartPar
En cliquant sur le bouton « validé » vous changer l’état de votre bordereau indiquant que celui\sphinxhyphen{}ci à été réellement crédité sur votre compte bancaire.
Si le \sphinxstyleemphasis{code bancaire} utilisé comporte un ensemble de 2 comptes (bancaire + attente) une écriture comptable sera alors générée afin de mettre à jour votre compte bancaire.


\section{Configuration}
\label{\detokenize{payoff/config:configuration}}\label{\detokenize{payoff/config::doc}}\begin{quote}

\sphinxAtStartPar
Menu \sphinxstyleemphasis{Administration/Modules (conf.)/Configuration du règlement}
\end{quote}

\sphinxAtStartPar
Ce menu vous permet de configurer les comptes courants et les moyens de paiement en usage dans votre structure.


\subsection{Code bancaire}
\label{\detokenize{payoff/config:code-bancaire}}
\sphinxAtStartPar
Dans cet écran, vous avez la possibilité d’enregistrer les références des différents comptes courants ouverts au nom de votre structure.
A l’aide du bouton « + Ajouter », vous pouvez en créer un nouveau et spécifier : la désignation, le RIB (référence), le code bancaire associé, le journal de mouvement désiré et optionnellement un compte d’attente.
Les compte doivent être déjà ouvert dans le plan comptable de l’exercice comme compte de trésorerie.

\sphinxAtStartPar
Dans le cas où vous préciseriez un compte d’attente, les mouvemets comptables générés seront un peu différent.
Cette option est à utiliser dans les règlements par chèque bancaire (ou assimilé comme des « bons » ou « tickets » subventionnés proposés par certaines collectivités locales).

\sphinxAtStartPar
En effet, au moment de la saisi de votre règlement, l’écriture sera alors créé sur le compte d’attente.
Vous serez alors invité d’utiliser l’outil de gestion de bordereaux de chèques et lors de la validation finale de celui\sphinxhyphen{}ci,
une écriture générale réalisera le mouvement financier entre ce compte d’attente et le compte bancaire proprement dit.


\subsection{Moyen de paiement}
\label{\detokenize{payoff/config:moyen-de-paiement}}
\sphinxAtStartPar
Vous pouvez ici préciser les moyens de paiement que vous acceptez.

\sphinxAtStartPar
Actuellement, 5 moyens de paiement sont possibles sous \sphinxstyleemphasis{Diacamma}, pour chacun d’entre eux, vous devez préciser les paramètres correspondants:
\begin{quote}
\begin{itemize}
\item {} 
\sphinxAtStartPar
Le virement bancaire

\item {} 
\sphinxAtStartPar
Le chèque

\item {} 
\sphinxAtStartPar
Le paiement PayPal

\end{itemize}

\sphinxAtStartPar
Dans le cas de PayPal, si votre \sphinxstyleemphasis{Diacamma} est accessible via le Net, le paiement de votre débiteur vous est notifié sous votre application et le règlement correspondant généré directement dans votre logiciel.
Il est conseillé de cocher le champ \sphinxstyleemphasis{avec contrôle}. Ainsi, le lien de paiement présenté dans un courriel s’assurera dans un premier temps que, sous votre \sphinxstyleemphasis{Diacamma}, l’élément financier est toujours d’actualité.
\begin{itemize}
\item {} 
\sphinxAtStartPar
Le règlement en ligne, via une adresse internet

\item {} 
\sphinxAtStartPar
Le paiement MoneticoPaiement

\end{itemize}

\sphinxAtStartPar
Pour MoneticoPaiement, la plateforme de paiement du Crédit Mutuel et du CIC (voir \sphinxurl{https://www.monetico-paiement.fr}), veuillez prendre contact avec votre agence bancaire pour faire ouvrir votre accès.
Et si vous \sphinxstyleemphasis{Diacamma} est accessible via le Net, vous devez déclarer à la plateforme une adresse de retour de type « \sphinxurl{https://mon.site.diacamma/diacamma.payoff/validationPaymentMoneticoPaiement} » ( »\sphinxurl{https://mon.site.diacamma} » étant l’adresse d’accès à votre serveur \sphinxstyleemphasis{Diacamma})
\end{quote}

\sphinxAtStartPar
Ces moyens de paiement peuvent être utilisés par vos débiteurs, via précision en bas de couriel, pour régler ce qu’ils vous doivent.


\subsection{Paramètres}
\label{\detokenize{payoff/config:parametres}}
\sphinxAtStartPar
4 Paramètres actuellement :
\begin{itemize}
\item {} 
\sphinxAtStartPar
compte de caisse : code comptable à mouvementer pour les règlements en espèces

\item {} 
\sphinxAtStartPar
compte de frais bancaires : code comptable devant être utilisé pour comptabiliser les frais bancaires liés aux règlements. Une ligne d’écriture est alors ajoutée directement à l’écriture comptable correspondante si le champ « compte de frais bancaires » est renseigné, ce qui permet de saisir le montant des frais.

\item {} 
\sphinxAtStartPar
sujet et message par défaut proposés à l’envoie des justificatifs d’un paiement.

\end{itemize}


\chapter{Lucterios contacts}
\label{\detokenize{contacts/index:lucterios-contacts}}\label{\detokenize{contacts/index::doc}}
\sphinxAtStartPar
Aide relative aux fonctionnalités de gestion de contacts moraux ou physiques.


\section{Les contact physiques (personnes physiques)}
\label{\detokenize{contacts/individual:les-contact-physiques-personnes-physiques}}\label{\detokenize{contacts/individual::doc}}
\sphinxAtStartPar
Un contact physique est une personne, homme ou femme, avec qui votre structure est en relation. Il peut s’agir d’un adhérent, d’un copropriétaire mais aussi d’un contact, personne physique, de l’un de vos interlocuteurs (exemple le commercial d’un de vos fournisseurs).
\begin{quote}

\sphinxAtStartPar
Menu \sphinxstyleemphasis{Bureautique/Adresses et Contacts/Personnes physiques}
\end{quote}


\subsection{Liste de vos contacts physiques}
\label{\detokenize{contacts/individual:liste-de-vos-contacts-physiques}}
\sphinxAtStartPar
La liste des personnes déjà enregistrées s’affiche. Étant donné que celle\sphinxhyphen{}ci peut devenir importante, il est possible de filtrer les personnes sur le nom.

\sphinxAtStartPar
Depuis cet écran, vous avez aussi la possibilité d’imprimer la liste des personnes et les étiquettes pour le courrier.

\noindent\sphinxincludegraphics{{ListIndividual}.png}


\subsection{Visualisation d’un contact physique}
\label{\detokenize{contacts/individual:visualisation-d-un-contact-physique}}
\sphinxAtStartPar
La liste des personnes physiques étant affichée, le bouton « Editer » ou un double\sphinxhyphen{}clic sur la ligne correspondante au contact, permettent de visualiser la fiche du contact.

\noindent\sphinxincludegraphics{{ShowIndividual}.png}

\sphinxAtStartPar
La fiche consultée peut être modifiée, bouton « modifier » et imprimée, bouton « Imprimer ».
Si cette personne n’est pas associée à d’autres enregistrements de l’application, vous avez la possibilité de supprimer sa fiche.

\sphinxAtStartPar
Vous pouvez également attribuer au contact physique un alias de connexion à l’application, assorti de droits et de permissions (voir Les utilisateurs).

\noindent\sphinxincludegraphics{{PermissionsIndividual}.png}


\subsection{Ajout d’un contact physique}
\label{\detokenize{contacts/individual:ajout-d-un-contact-physique}}
\sphinxAtStartPar
Depuis la liste précédente, vous avez aussi la possibilité d’ajouter une nouvelle personne à l’aide du bouton « + Créer ».

\noindent\sphinxincludegraphics{{EditIndividual}.png}


\subsection{Recherche d’un contact physique}
\label{\detokenize{contacts/individual:recherche-d-un-contact-physique}}\begin{quote}

\sphinxAtStartPar
Menu Bureautique/Adresses et Contacts/Recherche de personne physique
\end{quote}

\sphinxAtStartPar
Définissez les critères de recherche grâce à quoi seront extraites de la base toutes les fiches y satisfaisant.
Vous pourrez alors imprimer cette liste ou visualiser/modifier une fiche.

\noindent\sphinxincludegraphics{{FindIndividual}.png}

\sphinxAtStartPar
Les critères de filtre peuvent être sauvegardés pour une utilisation ultérieure.


\section{Les contacts moraux (personnes morales)}
\label{\detokenize{contacts/legal_entity:les-contacts-moraux-personnes-morales}}\label{\detokenize{contacts/legal_entity::doc}}
\sphinxAtStartPar
Entreprises, associations, établissements publics, …, les personnes morales sont des groupements de personnes dotées de la personnalité juridique. Elles peuvent être composées d’une ou de plusieurs personnes, individus ou personnes morales.
Pour chacune des personnes morales avec qui votre structure est en contact, une fiche peut être tenue sous \sphinxstyleemphasis{Diacamma}.
\begin{quote}

\sphinxAtStartPar
Menu \sphinxstyleemphasis{Bureautique/Adresses et Contacts/Personnes morales}
\end{quote}


\subsection{Liste de vos contacts moraux}
\label{\detokenize{contacts/legal_entity:liste-de-vos-contacts-moraux}}
\sphinxAtStartPar
Vous pouvez consulter la liste des structures pour lesquelles une fiche a déjà été enregistrée.
Chaque contact moral est associé à une catégorie grâce à quoi vous pouvez filtrer la liste des contacts sur le type de structures.

\noindent\sphinxincludegraphics{{ListLegalEntity}.png}

\sphinxAtStartPar
Depuis cet écran, vous avez aussi la possibilité d’imprimer la liste des structures avec le bouton « Liste » et pouvez imprimer les étiquettes pour le courrier.

\sphinxAtStartPar
Une nouvelle fiche est ouverte à l’aide du bouton « + Créer » et toute fiche peut être supprimée, à la condition de ne pas être associée à un autre enregistrement de votre base (exemple une facture saisie).


\subsection{Visualisation d’un contact moral}
\label{\detokenize{contacts/legal_entity:visualisation-d-un-contact-moral}}
\sphinxAtStartPar
Depuis la liste précédente, la fiche d’une structure peut être visualisée à l’aide du bouton « Editer » ou d’un double\sphinxhyphen{}clic sur la ligne correspondante au contact.

\noindent\sphinxincludegraphics{{ShowLegalEntity}.png}

\sphinxAtStartPar
Cette fiche peut ensuite être imprimée avec le bouton du même nom.


\subsection{Modifier un contact moral}
\label{\detokenize{contacts/legal_entity:modifier-un-contact-moral}}
\sphinxAtStartPar
La fiche étant toujours à l’écran, utilisez le bouton « Modifier » pour y apporter toute correction.

\noindent\sphinxincludegraphics{{EditLegalEntity}.png}


\subsection{Responsables d’un contact moral}
\label{\detokenize{contacts/legal_entity:responsables-d-un-contact-moral}}
\sphinxAtStartPar
Vous avez la possibilité d’associer une personne physique à un contact moral : onglet « Membres » et bouton « + Ajouter ».
Sélectionnez la personne physique. Si elle n’est pas répertorié dans votre base, vous avez la possibilité d’y pourvoir.
Tout nouveau membre peut être assorti d’une fonction.

\noindent\sphinxincludegraphics{{ResponsabilityLegalEntity}.png}


\subsection{Recherche d’un contact moral}
\label{\detokenize{contacts/legal_entity:recherche-d-un-contact-moral}}
\sphinxAtStartPar
Le menu \sphinxstyleemphasis{Bureautique/Adresses et Contacts/Recherche de personne morale} vous permet d’extraire de votre base les personnes morales satisfaisant aux critères saisis. Ces critères peuvent être sauvegardés pour une utilisation ultérieure.

\noindent\sphinxincludegraphics{{FindLegalEntity}.png}


\section{Configuration et paramétrage}
\label{\detokenize{contacts/configuration:configuration-et-parametrage}}\label{\detokenize{contacts/configuration::doc}}
\sphinxAtStartPar
Dans le menu \sphinxstyleemphasis{Administration/Modules (conf.)} vous avez à votre disposition des outils pour configurer la gestion des contacts.


\subsection{Configuration des contacts}
\label{\detokenize{contacts/configuration:configuration-des-contacts}}
\sphinxAtStartPar
Dans cet écran, vous avez la possibilité de créer ou de modifier la liste des fonctions et des responsabilités que peuvent avoir les contacts  associés aux personnes physiques et morales avec lesquelles vous êtes en relation. Vous pouvez aussi créer ou modifier les types de structures pour vous aider à classer vos personnes morales.

\sphinxAtStartPar
Il se peut que vous ayez besoin de préciser des informations supplémentaires pour vos différents contacts. Vous avez ici la possibilité d’ajouter des champs personnalisés pour chaque type de contacts. Pour ajouter un champ, vous devez simplement préciser le modèle de contact auquel il s’applique, indiquer son titre ainsi que définir son type et préciser si le champ est multiligne ou non.

\sphinxAtStartPar
Cinq types sont possibles :
\begin{itemize}
\item {} 
\sphinxAtStartPar
chaîne de texte

\item {} 
\sphinxAtStartPar
nombre entier

\item {} 
\sphinxAtStartPar
nombre à virgule (réel)

\item {} 
\sphinxAtStartPar
valeur Oui/Non (booléen)

\item {} 
\sphinxAtStartPar
sélection dans une liste (énumération)

\end{itemize}

\sphinxAtStartPar
Dans le cas de l’énumération, vous devez définir la liste des valeurs possibles (mots) séparées par un point\sphinxhyphen{}virgule.


\subsection{Codes postaux/villes}
\label{\detokenize{contacts/configuration:codes-postaux-villes}}
\sphinxAtStartPar
En saisie, l’outil va automatiquement rechercher la ville (ou les villes) associée(s) au code postal que vous entrerez afin de vous faciliter la saisie de vos contacts.
Dans cet écran, vous pouvez ajouter les codes postaux manquants.
Par défaut, les codes postaux français et suisses sont insérés.


\chapter{Lucterios messagerie}
\label{\detokenize{mailing/index:lucterios-messagerie}}\label{\detokenize{mailing/index::doc}}
\sphinxAtStartPar
Aide relative aux fonctionnalités de courier, de publipostage et de SMS.


\section{Publipostage}
\label{\detokenize{mailing/mailing:publipostage}}\label{\detokenize{mailing/mailing::doc}}\begin{quote}

\sphinxAtStartPar
Menu \sphinxstyleemphasis{Bureautique/Publipostage/Messages courriel}
\end{quote}


\subsection{Création d’un message}
\label{\detokenize{mailing/mailing:creation-d-un-message}}
\sphinxAtStartPar
Bouton « + Ajouter »

\sphinxAtStartPar
Une fois votre message rédigé, validez\sphinxhyphen{}le en cliquant sur « Ok ». En ouvrant l’onglet « Destinataires », vous allez pouvoir saisir le ou les critères permettant de filtrer vos contacts. Le résultat de la requête est affichée à l’écran. Vous pouvez l’affiner et sauvegarder vos critères pour une utilisation ultérieure.

\sphinxAtStartPar
Votre requête prête, cliquez sur « validée ». D’autres requêtes peuvent être ajoutées à la première, leurs résultats se cumulant.
Au moment où votre courrier sera généré, vos requêtes seront de nouveau exécutées, grâce à quoi votre base peut être mise à jour avant \sphinxstyleemphasis{Validation et transmission} des courriers si vous constatez qu’un contact est absent du résultat des requêtes.

\sphinxAtStartPar
Il est également possible de joindre à votre message un ou plusieurs documents sauvegardés dans le \sphinxstyleemphasis{Gestionnaire de documents}. Pour cela, ouvrez l’onglet « Documents ».

\sphinxAtStartPar
L’option \sphinxstyleemphasis{document(s) ajouté(s) via liens dans le message} permet d’ajouter un ensemble de liens de partage vers vos documents (et non plus comme pièces jointes). Cela permet la transmission de documents de taille importante ou qui risqueraient d’être supprimés par certains gestionnaires de courriel.

\noindent\sphinxincludegraphics{{mailing}.png}


\subsection{Validation \& transmission}
\label{\detokenize{mailing/mailing:validation-transmission}}\begin{description}
\item[{Une fois le message validé vous pouvez :}] \leavevmode\begin{itemize}
\item {} 
\sphinxAtStartPar
Soit générer une sortie PDF de l’ensemble des lettres à envoyer, personnalisées avec l’en\sphinxhyphen{}tête de chaque contact

\item {} 
\sphinxAtStartPar
Soit envoyer votre message par courriel, si celui\sphinxhyphen{}ci est correctement configuré. Bien sur, dans ce cas, seuls les contacts possédant une adresse électronique seront impactés par cet envoi.

\end{itemize}

\end{description}

\sphinxAtStartPar
De plus, dans le cas d’un envoi par courriel, vous pouvez consulter le rapport de transmission. Celui\sphinxhyphen{}ci vous indique les courriels envoyés et les éventuelles erreurs d’acheminement.

\sphinxAtStartPar
Si votre logiciel est accessible depuis internet, vous pouvez également consulter le nombre de fois que le destinataire a consulté ce message.
Ce mécanisme se base sur l’acceptation, par votre destinataire des images distantes présentent dans le message.

\noindent\sphinxincludegraphics{{transmission}.png}


\section{Envoi de SMS}
\label{\detokenize{mailing/sms:envoi-de-sms}}\label{\detokenize{mailing/sms::doc}}\begin{quote}

\sphinxAtStartPar
Menu \sphinxstyleemphasis{Bureautique/Publipostage/Messages SMS}
\end{quote}


\subsection{Création d’un message}
\label{\detokenize{mailing/sms:creation-d-un-message}}
\sphinxAtStartPar
L’envoi de SMS se fait via une interface proche de celle d’envoie de couriels.

\sphinxAtStartPar
Notons les différences:
\begin{itemize}
\item {} 
\sphinxAtStartPar
Spécificité de création de message
\begin{quote}

\sphinxAtStartPar
Vous devez précisez quel champ (tel1, tel2 ou un champ textuel personalisé) vous souhaitez utiliser pour envoyer vos messages.
Notez qu’il est possible d’envoyer à plusieurs numéro pour un même contact.
\end{quote}

\item {} 
\sphinxAtStartPar
La taille du message de celui\sphinxhyphen{}ci s’affiche.
\begin{quote}

\sphinxAtStartPar
Notez que certain caractère, comme les accents, peuvent compter double.
\end{quote}

\item {} 
\sphinxAtStartPar
Il n’est pas possible d’ajouter de document joint comme pour les courriels.

\end{itemize}


\subsection{Validation \& transmission}
\label{\detokenize{mailing/sms:validation-transmission}}
\sphinxAtStartPar
Une fois le message validé vous pouvez l’envoyer par SMS, si celui\sphinxhyphen{}ci est correctement configuré.
Bien sur, dans ce cas, seuls les contacts possédant une numéro de téléphone valide (voir configuration) seront impactés par cet envoi.

\sphinxAtStartPar
De plus, vous pouvez consulter le rapport de transmission.
Celui\sphinxhyphen{}ci vous indique les SMS envoyés et les éventuelles erreurs d’acheminement.


\section{Configuration de la messagerie}
\label{\detokenize{mailing/configuration:configuration-de-la-messagerie}}\label{\detokenize{mailing/configuration::doc}}\begin{quote}

\sphinxAtStartPar
Menu Administration/Modules (conf.)/Paramètres de courrier \& SMS
\end{quote}

\sphinxAtStartPar
Vous pouvez configurer ici des réglages pour l’envoi de couriel et de SMS.


\subsection{Configuration du couriel}
\label{\detokenize{mailing/configuration:configuration-du-couriel}}
\sphinxAtStartPar
Le serveur SMTP permettra au logiciel d’envoyer directement des messages à vos contacts.
Configurez donc ici les règlages de votre serveur.
Vous pouvez également préciser un \sphinxstyleemphasis{Fichier privé DKIM} et \sphinxstyleemphasis{Sélecteur DKIM} afin de signer vos envois de courriel.
Les paramètres \sphinxstyleemphasis{durée (en min) d’un lot de courriel} et \sphinxstyleemphasis{nombre de courriels par lot} sont utilisés pour l’envoie des messages en publipostage.

\sphinxAtStartPar
Un bouton \sphinxstyleemphasis{Envoyer} permet de tester vos règlages en envoyant un courriel de test à un destinataire choisi.
Il existe des outils permettant de vérifier si vos messages respectent des règles afin d’éviter d’être considéré comme des “pourriel”.
En autre, l’outil \sphinxurl{https://www.mail-tester.com} (gratuit jusqu’à 3 fois par jour) vous permet, en envoyant un message à l’adresse précisée, de vous établir une note de confiance.

\sphinxAtStartPar
Vous pouvez, entre autre, envoyer d’un nouveau mot de passe de connexion.
N’oubliez pas alors de préciser un petit message d’explication via le paramètre \sphinxstyleemphasis{Message de confirmation de connexion}.


\subsection{Configuration du SMS}
\label{\detokenize{mailing/configuration:configuration-du-sms}}
\sphinxAtStartPar
En configurant un fournisseur de SMS, vous pourrez alors envoyer des SMS à vos contacts.

\sphinxAtStartPar
Pour chaque fournisseur, vous devrez préciser le champ « Expression d’analyse de numéro (SMS) ».
Il correspond à une expression régulière afin de savoir comment transformer un numéro de téléphone en numéro international.
Par défaut, il correspond au numéro français: \sphinxstyleemphasis{\textasciicircum{}0({[}67{]}{[}0\sphinxhyphen{}9{]}\{8\})\$|+33\{0\}}
\begin{quote}

\sphinxAtStartPar
Il vérifie que le numéro comporte 10 chiffres.
Il commance par “06” ou “07”.
Il remplacera le “0” par “+33”.
\end{quote}

\sphinxAtStartPar
Suivant les besoins et les demandes, d’autres fournisseurs pourront être ajoutés à l’avenir.


\subsubsection{Mailjet SMS}
\label{\detokenize{mailing/configuration:mailjet-sms}}
\sphinxAtStartPar
Mailjet (\sphinxurl{https://www.mailjet.com/}) est une entreprise française qui propose, entre autres, d’envoyer des SMS.

\sphinxAtStartPar
Pour utiliser ce fournisseur, vous devez créer un compte sur leur site web.

\sphinxAtStartPar
Rendez vous ensuite sur leur portail SMS (\sphinxurl{https://app.mailjet.com/sms}) où vous pourrez configurer votre accès au SMS:
\begin{itemize}
\item {} 
\sphinxAtStartPar
Générer votre token d’accès

\item {} 
\sphinxAtStartPar
Créditer votre solde de SMS prépayé

\end{itemize}

\sphinxAtStartPar
Notez que le coùt d’un SMS dépend de sa taille et de sa destination.

\sphinxAtStartPar
Vous devrez ensuite indiquer dans « Options pour fournisseur SMS » votre « api token » généré précédement
ainsi qu’un « alias » qui sera l’identifiant présent dans le SMS envoyé.

\sphinxAtStartPar
Une fois votre configuration terminer, un bouton \sphinxstyleemphasis{Envoyer} permet de tester vos règlages en envoyant un SMS de contrôle à un numéro choisi.


\chapter{Lucterios documents}
\label{\detokenize{documents/index:lucterios-documents}}\label{\detokenize{documents/index::doc}}
\sphinxAtStartPar
Aide relative aux fonctionnalités de gestion documentaire.


\section{Gestion de fichiers}
\label{\detokenize{documents/shared_document:gestion-de-fichiers}}\label{\detokenize{documents/shared_document::doc}}

\subsection{Liste des documents}
\label{\detokenize{documents/shared_document:liste-des-documents}}
\sphinxAtStartPar
Pour retrouver plus aisément vos documents, sous \sphinxstyleemphasis{Diacamma}, ceux\sphinxhyphen{}ci peuvent être enregistrés dans des dossiers et des sous\sphinxhyphen{}dossiers du gestionnaire de documents.
Chaque dossier est assorti d’une description et d’informations relatives à la dernière modification.
\begin{quote}

\sphinxAtStartPar
Menu \sphinxstyleemphasis{Bureautique/Gestion de fichiers et de documents/Documents}
\end{quote}

\sphinxAtStartPar
Le retour au dossier\sphinxhyphen{}parent est possible grâce au bouton “\textless{}”.
Si vous disposez des droits, le bouton d’édition situé à coté permet d’éditer les propriétés du dossier actif et de les modifier.

\noindent\sphinxincludegraphics{{listdoc}.png}

\sphinxAtStartPar
Suivant vos permissions, la zone de gauche vous permet de gérer vos dossiers.
En utilisant le bouton “+”, il est possible de créer un nouveau dossier.
Après l’avoir sélectionné, vous pouvez aussi en supprimer un et son contenu, à l’aide du bouton “\sphinxhyphen{}“.
Affichez le contenu d’un dossier après l’avoir sélectionné et avoir cliqué sur le bouton “\textgreater{}” ou faites un double\sphinxhyphen{}clic sur la ligne correspondante au dossier.

\sphinxAtStartPar
Depuis la zone centrale, vous pouvez visualiser les documents contenus dans le dossier courant.
Suivant vos permissions, vous pouvez ajouter ou supprimer un document.
\sphinxstyleemphasis{Diacamma} mémorise l’utilisateur et la date de création de tout document ainsi que les informations relatives à la dernière modification.

\sphinxAtStartPar
Affichez la fiche d’un document à l’aide d’un double\sphinxhyphen{}clic sur son nom.

\noindent\sphinxincludegraphics{{showdoc}.png}
\begin{description}
\item[{Depuis cette fiche, il vous est possible :}] \leavevmode\begin{itemize}
\item {} 
\sphinxAtStartPar
de le modifier en l’important de nouveau

\item {} 
\sphinxAtStartPar
de générer un lien de téléchargement. Ce lien web peut être transmis à une personne n’ayant aucun droit d’accès à votre logiciel afin qu’elle puisse télécharger le document

\end{itemize}

\end{description}

\sphinxAtStartPar
\sphinxstylestrong{Attention:} Votre instance doit être accessible sur internet pour que ce lien puisse fonctionner.


\subsection{Recherche de documents}
\label{\detokenize{documents/shared_document:recherche-de-documents}}\begin{quote}

\sphinxAtStartPar
Menu \sphinxstyleemphasis{Bureautique/Gestion documentaire/Recherche de document}
\end{quote}

\sphinxAtStartPar
Saisissez les critères de recherche de documents et validez. Ces critères sont sauvegardables pour une utilisation ultérieure.

\sphinxAtStartPar
Diacamma parcourera tous les dossiers du gestionnaire de documents afin d’en extraire la liste de ceux satisfaisant aux critères saisis.


\section{Editeur de documents}
\label{\detokenize{documents/editor:editeur-de-documents}}\label{\detokenize{documents/editor::doc}}
\sphinxAtStartPar
Il est possible de configurer l’outil afin de pouvoir éditer certains documents directement via l’interface « en ligne ».

\sphinxAtStartPar
Des outils d’édition, libres et gratuits, sont actuellement configurables afin de les utiliser pour consulter et modifier des documents.

\sphinxAtStartPar
\sphinxstylestrong{Note :} Ces outils sont gérés par des équipes complètement différentes, il se peut que certains de leurs comportements ne correspondent pas à vos attentes.


\subsection{Etherpad}
\label{\detokenize{documents/editor:etherpad}}
\sphinxAtStartPar
Éditeur pour document textuel.

\sphinxAtStartPar
Site Web
\begin{quote}

\sphinxAtStartPar
\sphinxurl{https://etherpad.org/}
\end{quote}

\sphinxAtStartPar
Installation
\begin{quote}

\sphinxAtStartPar
Le tutoriel de framasoft explique bien comment l’installer: \sphinxurl{https://framacloud.org/fr/cultiver-son-jardin/etherpad.html}
\end{quote}

\sphinxAtStartPar
Configurer
\begin{quote}

\sphinxAtStartPar
Éditer le fichier « settings.py » contenu dans le répertoire de votre instance.
Ajouter et adapter la ligne ci\sphinxhyphen{}dessous:
\begin{itemize}
\item {} 
\sphinxAtStartPar
url : adresse d’accès d’Etherpad

\item {} 
\sphinxAtStartPar
apikey : contenu de la clef de sécurité (fichier APIKEY.txt contenu dans l’installation d’etherpad)

\end{itemize}
\end{quote}

\begin{sphinxVerbatim}[commandchars=\\\{\}]
\PYG{c+c1}{\PYGZsh{} extra}
\PYG{n}{ETHERPAD} \PYG{o}{=} \PYG{p}{\PYGZob{}}\PYG{l+s+s1}{\PYGZsq{}}\PYG{l+s+s1}{url}\PYG{l+s+s1}{\PYGZsq{}}\PYG{p}{:} \PYG{l+s+s1}{\PYGZsq{}}\PYG{l+s+s1}{http://localhost:9001}\PYG{l+s+s1}{\PYGZsq{}}\PYG{p}{,} \PYG{l+s+s1}{\PYGZsq{}}\PYG{l+s+s1}{apikey}\PYG{l+s+s1}{\PYGZsq{}}\PYG{p}{:} \PYG{l+s+s1}{\PYGZsq{}}\PYG{l+s+s1}{jfks5dsdS65lfGHsdSDQ4fsdDG4lklsdq6Gfs4Gsdfos8fs}\PYG{l+s+s1}{\PYGZsq{}}\PYG{p}{\PYGZcb{}}
\end{sphinxVerbatim}

\sphinxAtStartPar
Usage
\begin{quote}
\begin{description}
\item[{Dans le gestionnaire de documents, vous avez plusieurs actions qui apparaissent alors:}] \leavevmode\begin{itemize}
\item {} 
\sphinxAtStartPar
Un bouton « + Fichier » vous permettant de créer un document txt ou html depuis la liste d’un dossier.

\item {} 
\sphinxAtStartPar
Un bouton « Editeur » pour ouvrir l’éditeur Etherpad depuis la fiche du document.

\end{itemize}

\end{description}
\end{quote}

\noindent\sphinxincludegraphics{{etherpad}.png}


\subsection{Ethercalc}
\label{\detokenize{documents/editor:ethercalc}}
\sphinxAtStartPar
Éditeur pour tableau de calcul.

\sphinxAtStartPar
Site Web
\begin{quote}

\sphinxAtStartPar
\sphinxurl{https://ethercalc.net/}
\end{quote}

\sphinxAtStartPar
Installation
\begin{quote}

\sphinxAtStartPar
Sur le site de l’éditeur, une petit explication indique comment l’installer.
\end{quote}

\sphinxAtStartPar
Configurer
\begin{quote}

\sphinxAtStartPar
Editer le fichier « settings.py » contenu dans le répertoire de votre instance.
Ajouter et adapter la ligne ci\sphinxhyphen{}dessous:
\begin{itemize}
\item {} 
\sphinxAtStartPar
url : adresse d’accès d’Ethercal

\end{itemize}
\end{quote}

\begin{sphinxVerbatim}[commandchars=\\\{\}]
\PYG{c+c1}{\PYGZsh{} extra}
\PYG{n}{ETHERCALC} \PYG{o}{=} \PYG{p}{\PYGZob{}}\PYG{l+s+s1}{\PYGZsq{}}\PYG{l+s+s1}{url}\PYG{l+s+s1}{\PYGZsq{}}\PYG{p}{:} \PYG{l+s+s1}{\PYGZsq{}}\PYG{l+s+s1}{http://localhost:8000}\PYG{l+s+s1}{\PYGZsq{}}\PYG{p}{\PYGZcb{}}
\end{sphinxVerbatim}

\sphinxAtStartPar
Usage
\begin{quote}
\begin{description}
\item[{Dans le gestionnaire de documents, vous avez plusieurs actions qui apparaissent alors:}] \leavevmode\begin{itemize}
\item {} 
\sphinxAtStartPar
Un bouton « + Fichier » vous permettant de créer un document csv, ods ou xmlx depuis la liste d’un dossier.

\item {} 
\sphinxAtStartPar
Un bouton « Editeur » pour ouvrir l’éditeur Ethercalc depuis la fiche du document.

\end{itemize}

\end{description}
\end{quote}

\noindent\sphinxincludegraphics{{ethercalc}.png}


\subsection{OnlyOffice Docs Community Edition}
\label{\detokenize{documents/editor:onlyoffice-docs-community-edition}}
\sphinxAtStartPar
Éditeurs en ligne pour les documents texte, les feuilles de calcul et les présentations.

\sphinxAtStartPar
Site Web
\begin{quote}

\sphinxAtStartPar
\sphinxurl{https://www.onlyoffice.com/fr/office-suite.aspx}
\end{quote}

\sphinxAtStartPar
Installation
\begin{quote}

\sphinxAtStartPar
Nous vous recommandons l’installation via Docker comme expliqué ici:
\sphinxurl{https://helpcenter.onlyoffice.com/installation/docs-community-install-docker.aspx}
\end{quote}

\sphinxAtStartPar
Configurer
\begin{quote}

\sphinxAtStartPar
Editer le fichier « settings.py » contenu dans le répertoire de votre instance.
Ajouter et adapter la ligne ci\sphinxhyphen{}dessous:
\begin{itemize}
\item {} 
\sphinxAtStartPar
url : adresse d’accès d’OnlyOffice

\end{itemize}
\end{quote}

\begin{sphinxVerbatim}[commandchars=\\\{\}]
\PYG{c+c1}{\PYGZsh{} extra}
\PYG{n}{ONLYOFFICE} \PYG{o}{=} \PYG{p}{\PYGZob{}}\PYG{l+s+s1}{\PYGZsq{}}\PYG{l+s+s1}{url}\PYG{l+s+s1}{\PYGZsq{}}\PYG{p}{:} \PYG{l+s+s1}{\PYGZsq{}}\PYG{l+s+s1}{http://localhost:8100}\PYG{l+s+s1}{\PYGZsq{}}\PYG{p}{\PYGZcb{}}
\end{sphinxVerbatim}

\sphinxAtStartPar
Usage
\begin{quote}
\begin{description}
\item[{Dans le gestionnaire de documents, vous avez plusieurs actions qui apparaissent alors:}] \leavevmode\begin{itemize}
\item {} 
\sphinxAtStartPar
Un bouton « + Fichier » vous permettant de créer un document csv, xlsx, ods, docx, odt, txt, pptx ou odp depuis la liste d’un dossier.

\item {} 
\sphinxAtStartPar
Un bouton « Editeur » pour ouvrir l’éditeur OnlyOffice depuis la fiche du document.

\item {} 
\sphinxAtStartPar
A noter que vous avez également la possibilité de visualiser, en lecture seule, les documents xls, doc, ppt ou pdf

\end{itemize}

\end{description}
\end{quote}

\noindent\sphinxincludegraphics{{onlyoffice}.png}


\section{Configurer les dossiers}
\label{\detokenize{documents/configuration:configurer-les-dossiers}}\label{\detokenize{documents/configuration::doc}}
\sphinxAtStartPar
Pour votre gestion documentaire, vous disposez d’un ensemble d’outils.
\begin{quote}

\sphinxAtStartPar
Menu \sphinxstyleemphasis{Administration/Module (conf)/Dossiers}
\end{quote}

\sphinxAtStartPar
A l’écran la liste des dossiers existants s’affiche. Vous avez la possibilité d’en créer de nouveaux et de modifier les paramètres des dossiers déjà présents.

\noindent\sphinxincludegraphics{{configuration}.png}

\sphinxAtStartPar
En définissant judicieusement un dossier comme sous\sphinxhyphen{}dossier d’un dossier\sphinxhyphen{}parent, vous pouvez  mettre en place une arborescence de dossiers respectueuse de votre plan de classement.

\sphinxAtStartPar
Pour chaque dossier, vous pouvez aussi définir les droits des groupes d’utilisateurs, que cela soit pour la visualisation ou la modification des fichiers qui sont/seront enregistrés dans le dossier.
Ainsi, les utilisateurs appartenant aux groupes de visualisation pourront seulement consulter les documents contenus dans ce dossier. Les utilisateurs appartenant aux groupes de modification pourront, eux, les modifier ou les supprimer.

\sphinxAtStartPar
Le bouton « Extraire » permet de réaliser la sauvegarde d’un dossier et de son contenu sous forme d’une archive au format zip.
Quand au bouton « Importer », il permet d’importer une archive au format zip qui sera décompressée automatiquement dans le dossier de destination spécifié.


\chapter{Coeur Lucterios}
\label{\detokenize{CORE/index:coeur-lucterios}}\label{\detokenize{CORE/index::doc}}
\sphinxAtStartPar
Aide relative aux fonctionnalités générales de cet outil de gestion.


\section{Mot de passe}
\label{\detokenize{CORE/password:mot-de-passe}}\label{\detokenize{CORE/password::doc}}\begin{quote}

\sphinxAtStartPar
Menu \sphinxstyleemphasis{Général/Mot de passe}
\end{quote}

\sphinxAtStartPar
Vous pouvez changer le mot de passe d’accès de l’utilisateur courant.

\noindent\sphinxincludegraphics{{password}.png}

\sphinxAtStartPar
Pour plus de sécurité, nous vous conseillons d’utiliser un mot de passe comprenant des lettres et des chiffres et ne constituant pas un mot compréhensible.


\section{Les groupes}
\label{\detokenize{CORE/groups:les-groupes}}\label{\detokenize{CORE/groups::doc}}\begin{quote}

\sphinxAtStartPar
Menu \sphinxstyleemphasis{Administration/Gestion des Droits/Les groupes}
\end{quote}

\sphinxAtStartPar
Créez, modifiez ou supprimez un groupe de droits.

\noindent\sphinxincludegraphics{{group}.png}

\sphinxAtStartPar
Un groupe de droits permet de définir les autorisations (accès à certaines fonctionnalités du logiciel) qui sont consenties aux utilisateurs de l’application rattachés à celui\sphinxhyphen{}ci.

\noindent\sphinxincludegraphics{{group_modify}.png}


\section{Les utilisateurs}
\label{\detokenize{CORE/users:les-utilisateurs}}\label{\detokenize{CORE/users::doc}}\begin{quote}

\sphinxAtStartPar
Menu \sphinxstyleemphasis{Administration/Gestion des Droits/Les utilisateurs}
\end{quote}

\sphinxAtStartPar
Créez, modifiez ou désactivez un utilisateur de l’application. Chacun bénéficie d’un droit de connexion à \sphinxstyleemphasis{diacamma} dont vous définissez l’étendue.

\noindent\sphinxincludegraphics{{users}.png}

\sphinxAtStartPar
Depuis cette liste, vous pouvez créer ou modifier un utilisateur : son alias, son nom et son mot de passe.

\noindent\sphinxincludegraphics{{user_info}.png}

\sphinxAtStartPar
Vous pouvez aussi l’inscrire dans des groupes, lui accorder des permissions suplémentaires afin de définir son niveau d’accès au logiciel. Vous pouvez aussi désactiver un utilisateur pour lui interdire l’accès à l’application.

\noindent\sphinxincludegraphics{{user_permissions}.png}


\section{L’architecture du logiciel}
\label{\detokenize{CORE/architecture:l-architecture-du-logiciel}}\label{\detokenize{CORE/architecture::doc}}
\sphinxAtStartPar
Depuis le commencement de ce logiciel, les développeurs ont voulu que cette application puisse avoir une architecture ouverte permettant des évolutions les plus larges.



\renewcommand{\indexname}{Index}
\printindex
\end{document}